\documentclass[12pt]{article}
% \documentstyle{iascars2017}

% \usepackage{iascars2017}

\pagestyle{myheadings}
\pagenumbering{arabic}
\topmargin 0pt \headheight 23pt \headsep 24.66pt
%\topmargin 11pt \headheight 12pt \headsep 13.66pt
\parindent = 3mm


\begin{document}


\begin{flushleft}


{\LARGE\bf  Consistency of Linear Mixed-Effects Model Selection with Inconsistent Covariance Parameter Estimators}


\vspace{1.0cm}

ChihHao Chang$^1$, HsinCheng Huang$^2$ and ChingKang Ing$^3$

\begin{description}

\item $^1 \;$ Institute of Statistics, National University of Kaohsiung, Taiwan

\item $^2 \;$ Institute of Statistical Science, Academia Sinica, Taiwan
\item $^3 \;$ Institute of Statistics, National Tsing Hua Univeristy, Taiwan

\end{description}

\end{flushleft}

%  ***** ADD ENOUGH VERTICAL SPACE HERE TO ENSURE THAT THE *****
%  ***** ABSTRACT (OR MAIN TEXT) STARTS 5 CM BELOW THE TOP *****

\vspace{0.4cm}

\noindent {\bf Abstract}.
For linear mixed-effects models with data collected within one
cluster, the maximum likelihood estimators of covariance
parameters cannot be estimated consistently. Hence the asymptotic
behaviors of likelihood-based information criteria, such as Akaike's
information criterion (AIC) are rarely discussed in literature. In the contrast,
the number of the clusters is generally assumed going to infinity
with the sample size to guarantee the consistency of the covariance
parameter estimators and thereby guarantees the consistency of the
model selection procedures. In this talk, under some mild
conditions, we establish asymptotic theorems for ML estimators of
covariance parameters when the number of clusters is fixed.
Furthermore, the asymptotic behaviors of the generalized information
criterion, which includes AIC as special cases, are
well studied in our research.

\vskip 2mm

%\section{ First-level heading}
%The C98 head 1 style leaves a half-line spacing below a
%first-level heading. There should be one blank line above
%a first-level heading.
%
%\subsection { Second-level heading}
%There should also be one blank line above a second- or
%third-level heading (but no extra space below them).
%
%Do not intent the first paragraph following a heading.
%Second and subsequent paragraphs are indented by one Tab
%character (= 3 mm). If footnotes are used, they should be
%placed at the foot of the page\footnote{ Footnotes are separated
%from the text by a blank line and a printed line of length 3.5 cm.
%They should be printed in 9-point Times Roman in single line spacing.}.
%
%\subsubsection { Third-level heading}
%Please specify references using the conventions
%illustrated below. Each should begin on a new line, and
%second and subsequent lines should be on the same page
%indented by 3 mm.

\subsection*{References}

\begin{description}
\item 
Fan, Y. and Li, R. (2012).
Variable selection in linear mixed effects models.
In: \textit{The Annals of Statistics}, \textbf{40}, 2043 - 2068.
\item
Jiang, J.,~Rao,~J.~S.~Gu, Z. and Nguyen, T.~(2008). 
Fence methods for mixed model selection.
In: \textit{The Annals of Statistics}, \textbf{36}, 1669-1692.
\item
M\"{u}ller, S., Scealy, J. L.~and Welsh, A. H.~(2013). 
Model Selection in Linear Mixed Models.
In: \textit{Statistical Science}, \textbf{28}, 135-167.
\item
Sun, Y., Zhang, W. and Tong, H.~(2007). 
Estimation of the covariance matrix of random effects in longitudinal studies.
In: \textit{The Annals of Statistics}, \textbf{35}, 2795-2814.
\end{description}

\end{document}





