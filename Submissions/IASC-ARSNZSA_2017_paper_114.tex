\documentclass[12pt]{article}
% \documentstyle{iascars2017}

% \usepackage{iascars2017}

\pagestyle{myheadings} 
\pagenumbering{arabic}
\topmargin 0pt \headheight 23pt \headsep 24.66pt
%\topmargin 11pt \headheight 12pt \headsep 13.66pt
\parindent = 3mm 


\begin{document}


\begin{flushleft}


{\LARGE\bf A new approach to distribution free tests in contingency tables}


\vspace{1.0cm}

Thuong T. M. Nguyen$^1$ 

\begin{description}

\item $^1$ School of Mathematics and Statistics, Victoria University of Wellington, Wellington, 6140, New Zealand

\end{description}

\end{flushleft}

%  ***** ADD ENOUGH VERTICAL SPACE HERE TO ENSURE THAT THE *****
%  ***** ABSTRACT (OR MAIN TEXT) STARTS 5 CM BELOW THE TOP *****

\vspace{0.75cm}

\noindent {\bf Abstract}. We will discuss in this talk a new construction for a class of distribution free goodness of fit tests for the classical problem: testing independence in contingency tables. 
The point is that this problem has been stayed with only one asymptotically distribution free goodness of fit test for a long time, the chi-square test. We will show that our class 
of new distribution free goodness of fit tests is very wide and discuss the cases where the new tests perform better than the conventional chi-square test.
\vskip 2mm

\noindent {\bf Keywords}.
Contingency tables, distribution free, goodness of fit tests


%\section{ First-level heading}
%The C98 head 1 style leaves a half-line spacing below a
%first-level heading. There should be one blank line above
%a first-level heading.
%        
%\subsection { Second-level heading}
%There should also be one blank line above a second- or
%third-level heading (but no extra space below them).
%
%Do not intent the first paragraph following a heading.
%Second and subsequent paragraphs are indented by one Tab
%character (= 3 mm). If footnotes are used, they should be
%placed at the foot of the page\footnote{ Footnotes are separated
%from the text by a blank line and a printed line of length 3.5 cm.
%They should be printed in 9-point Times Roman in single line spacing.}.
%        
%\subsubsection { Third-level heading}
%Please specify references using the conventions
%illustrated below. Each should begin on a new line, and
%second and subsequent lines should be on the same page
%indented by 3 mm.

\subsection*{References}

\begin{description}

\item
Khmaladze, E., (2013). Note on distribution free testing for discrete distribution, \textit{Annals of Statistics}, \textbf{41}, 2979--2993

\item
Khmaladze, E., (2016). Unitary transformations, empirical processes and distribution free testing, \textit{Bernoulli}, \textbf{22}, 563--588


\item Nguyen, T.T.M., (2017). A new approach to distribution free tests in contingency tables, \textit{Metrika}, \textbf{80}, 153--170

\end{description}

\end{document}





