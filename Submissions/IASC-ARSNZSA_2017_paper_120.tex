\documentclass[12pt]{article}
% \documentstyle{iascars2017}

% \usepackage{iascars2017}

\pagestyle{myheadings}
\pagenumbering{arabic}
\topmargin 0pt \headheight 23pt \headsep 24.66pt
%\topmargin 11pt \headheight 12pt \headsep 13.66pt
\parindent = 3mm


\begin{document}


\begin{flushleft}


{\LARGE\bf Forward selection in regression models based on robust estimation}


\vspace{1.0cm}

Shan Luo$^1$ and Zehua Chen$^2$

\begin{description}

\item $^1 \;$ Department of Statistics, School of Mathematical Sciences,
Shanghai Jiao Tong University, Shanghai 200240, China

\item $^2 \;$ Department of Statistics and Applied Probability,
National University of Singapore, Singapore 117546, Singapore

\end{description}

\end{flushleft}

%  ***** ADD ENOUGH VERTICAL SPACE HERE TO ENSURE THAT THE *****
%  ***** ABSTRACT (OR MAIN TEXT) STARTS 5 CM BELOW THE TOP *****

\vspace{0.75cm}

\noindent {\bf Abstract}. For the purpose of feature selection in ultra-high dimensional regression models, it is required that a sequence of candidate models and a criterion to select the ``best'' model from them are available. Under different scenarios, various methods have been proposed to achieve these two goals. Intuitively, it is straightforward to choose appropriate loss and penalty functions in a regularization method to accommodate specific characteristics of the data. However, the computation could be expensive for certain cases. From recent studies, we can see that sequential method is promising to produce good candidate models for ultra-high dimensional data. Moreover, it can be easily extended to complex models other than the linear regression model. In this paper, we propose a new feature selection method based on robust estimation.

\vskip 2mm

\noindent {\bf Keywords}.
Feature selection, robust estimation, sequential method.


%\section{ First-level heading}
%The C98 head 1 style leaves a half-line spacing below a
%first-level heading. There should be one blank line above
%a first-level heading.
%
%\subsection { Second-level heading}
%There should also be one blank line above a second- or
%third-level heading (but no extra space below them).
%
%Do not intent the first paragraph following a heading.
%Second and subsequent paragraphs are indented by one Tab
%character (= 3 mm). If footnotes are used, they should be
%placed at the foot of the page\footnote{ Footnotes are separated
%from the text by a blank line and a printed line of length 3.5 cm.
%They should be printed in 9-point Times Roman in single line spacing.}.
%
%\subsubsection { Third-level heading}
%Please specify references using the conventions
%illustrated below. Each should begin on a new line, and
%second and subsequent lines should be on the same page
%indented by 3 mm.

%\subsection*{References}
%
%\begin{description}
%
%\item
%Barnett, J.A., Payne, R.W. and Yarrow, D. (1990).
%\textit{Yeasts: Characteristics and identification: Second Edition.}
%Cambridge: Cambridge University Press.
%
%\item
%(ed.) Barnett, V., Payne, R. and Steiner, R. (1995).
%\textit{Agricultural Sustainability: Economic, Environmental and
%Statistical Considerations}. Chichester: Wiley.
%
%\item
%Payne, R.W. (1997).
%\textit{Algorithm AS314 Inversion of matrices Statistics},
%\textbf{46}, 295--298.
%
%\item
%Payne, R.W. and Welham, S.J. (1990).
%A comparison of algorithms for combination of information in generally
%balanced designs.
%In: \textit{COMPSTAT90 Proceedings in Computational Statistics}, 297--302.
%Heidelberg: Physica-Verlag.
%
%\end{description}

\end{document}





