\documentclass[12pt]{article}
% \documentstyle{iascars2017}

% \usepackage{iascars2017}

\pagestyle{myheadings}
\pagenumbering{arabic}
\topmargin 0pt \headheight 23pt \headsep 24.66pt
%\topmargin 11pt \headheight 12pt \headsep 13.66pt
\parindent = 3mm


\begin{document}


\begin{flushleft}


{\LARGE\bf Adaptive False Discovery Rate regression with
application in integrative analysis of large-scale
genomic data}


\vspace{1.0cm}

Can YANG$^1$

\begin{description}

\item $^1 \;$ Department of Mathematics, The Hong Kong University of Science and Techonolgy,
Clear Water Bay, Hong Kong.

\end{description}

\end{flushleft}

%  ***** ADD ENOUGH VERTICAL SPACE HERE TO ENSURE THAT THE *****
%  ***** ABSTRACT (OR MAIN TEXT) STARTS 5 CM BELOW THE TOP *****

\vspace{0.75cm}

\noindent {\bf Abstract}. Recent international projects, such as the Encyclopedia of DNA Elements (ENCODE) project, the Roadmap project and the Genotype-Tissue Expression (GTEx) project, have generated vast amounts of genomic annotation data, e.g., epigenome and transcriptome. There is great demanding of effective statistical approaches to integrate genomic annotations with the results from genome-wide association studies. In this talk, we introduce a statistical framework, named AdaFDR, for integrating multiple annotations to characterize functional roles of genetic variants that underlie human complex phenotypes. For a given phenotype, AdaFDR can adaptively incorporates relevant annotations for prioritization of genetic risk variants, allowing nonlinear effects among these annotations, such as interaction effects between genomic features. Specifically, we assume that the prior probability of a variant associated with the phenotype is a function of its annotations $F(X)$, where $X$ is the collection of the annotation status and $F(X)$ is an ensemble of decision trees, i.e., $F(X) = \sum_k f_k(X)$ and $f_k(X)$ is a shallow decision tree. We have developed an efficient EM-Boosting algorithm for model fitting, where a shallow decision tree grows in a gradient-Boosting manner (Friedman J. 2001) at each EM-iteration. Our framework inherits the nice property of gradient boosted trees: (1) The gradient accent property of the Boosting algorithm naturally guarantees the convergence of our EM-Boosting algorithm. (2) Based on the fitted ensemble $\hat{F}(X)$, we are able to rank the importance of annotations, measure the interaction among annotations and visualize the model via partial plots (Friedman J. 2008). Using AdaFDR, we performed integrative analysis of genome-wide association studies on human complex phenotypes and genome-wide annotation resources, e.g., Roadmap epigenome. The analysis results revealed interesting regulatory patterns of risk variants. These findings deepen our understanding of genetic architectures of complex phenotypes. The statistical framework developed here is also broadly applicable to many other areas for integrative analysis of rich data sets.

\vskip 2mm

\noindent {\bf Keywords}.
False Discovery Rate, integrative analysis, functional annotation, genomic data


%\section{ First-level heading}
%The C98 head 1 style leaves a half-line spacing below a
%first-level heading. There should be one blank line above
%a first-level heading.
%
%\subsection { Second-level heading}
%There should also be one blank line above a second- or
%third-level heading (but no extra space below them).
%
%Do not intent the first paragraph following a heading.
%Second and subsequent paragraphs are indented by one Tab
%character (= 3 mm). If footnotes are used, they should be
%placed at the foot of the page\footnote{ Footnotes are separated
%from the text by a blank line and a printed line of length 3.5 cm.
%They should be printed in 9-point Times Roman in single line spacing.}.
%
%\subsubsection { Third-level heading}
%Please specify references using the conventions
%illustrated below. Each should begin on a new line, and
%second and subsequent lines should be on the same page
%indented by 3 mm.

\subsection*{References}

\begin{description}


\item
Friedman, Jerome H (2001).
Greedy function approximation: a gradient boosting machine,
\textit{Annals of statistics},
\textbf{29:5},1189--1232.


\item Jerome H. Friedman and Bogdan E. Popescu (2008)
Predictive Learning via Rule Ensembles
\textit{The Annals of Applied Statistics},
\textbf{2:3}, 916--954


\end{description}

\end{document}





