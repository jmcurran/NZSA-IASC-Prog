\documentclass[12pt]{article}
% \documentstyle{iascars2017}

% \usepackage{iascars2017}

\pagestyle{myheadings} 
\pagenumbering{arabic}
\topmargin 0pt \headheight 23pt \headsep 24.66pt
%\topmargin 11pt \headheight 12pt \headsep 13.66pt
\parindent = 3mm 


\begin{document}


\begin{flushleft}


{\LARGE\bf Bayesian semi-parametric approaches to analyzing temperature trends}


\vspace{1.0cm}

Matthew Edwards$^1$, Claudia Kirch$^2$, Alexander Meier$^2$, Renate Meyer$^1$ and Blake Seers$^1$

\begin{description}

\item $^1 \;$ Department of Statistics, University of Auckland,
Auckland, New Zealand

\item $^2 \;$  Otto-von-Guericke University, Magdeburg, Germany

\end{description}

\end{flushleft}

%  ***** ADD ENOUGH VERTICAL SPACE HERE TO ENSURE THAT THE *****
%  ***** ABSTRACT (OR MAIN TEXT) STARTS 5 CM BELOW THE TOP *****

\vspace{0.75cm}

\noindent {\bf Abstract}. 
To discern potential impacts of global warming for New Zealand, it is imperative to accurately estimate long-term temperature trends. Here we analyse annual average temperature time series measured at several locations in New Zealand. The data are from a set of climate stations with no significant site changes since the 1930s. We specify a hierarchical semi-parametric model with a linear time and changepoints at known time points where re-siting of recorders or changes in instrumentation occurred. The power spectral density of the errors is estimated using a Bayesian nonparametric approach based on Whittle’s likelihood.  We use a nonparametric Bernstein polynomial prior on the spectral density with weights induced by a Dirichlet process. 
Posterior computation is implemented using a MH-within-Gibbs algorithm. We compare our results to those obtained by OLS and GLS.

\vskip 2mm

\noindent {\bf Keywords}.
stationary time series, spectral density, Bayesian nonparametrics, posterior consistency, Whittle likelihood, Bernstein-Dirichlet prior


%\section{ First-level heading}
%The C98 head 1 style leaves a half-line spacing below a
%first-level heading. There should be one blank line above
%a first-level heading.
%        
%\subsection { Second-level heading}
%There should also be one blank line above a second- or
%third-level heading (but no extra space below them).
%
%Do not intent the first paragraph following a heading.
%Second and subsequent paragraphs are indented by one Tab
%character (= 3 mm). If footnotes are used, they should be
%placed at the foot of the page\footnote{ Footnotes are separated
%from the text by a blank line and a printed line of length 3.5 cm.
%They should be printed in 9-point Times Roman in single line spacing.}.
%        
%\subsubsection { Third-level heading}
%Please specify references using the conventions
%illustrated below. Each should begin on a new line, and
%second and subsequent lines should be on the same page
%indented by 3 mm.

\subsection*{References}

\begin{description}


\item Kirch, C., Edwards, M., Meier, A., and Meyer, R. (2017). \textit{Beyond Whittle: Nonparametric correction of a parametric likelihood with a focus on Bayesian time series analysis},
{\tt  https://arxiv.org/abs/1701.04846}.

\item Thomson, P., Mullan, B. and Stuart, S. (2014). Estimating the slope and standard error of a long-term linear trend fitted to adjusted annual temperatures. Joint  NZSA/ORSNZ Conference, Wellington.


\end{description}

\end{document}





