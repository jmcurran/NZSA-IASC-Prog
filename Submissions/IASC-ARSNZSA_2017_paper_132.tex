\documentclass[12pt]{article}
% \documentstyle{iascars2017}

% \usepackage{iascars2017}

\pagestyle{myheadings} 
\pagenumbering{arabic}
\topmargin 0pt \headheight 23pt \headsep 24.66pt
%\topmargin 11pt \headheight 12pt \headsep 13.66pt
\parindent = 3mm 


\begin{document}


\begin{flushleft}


{\LARGE\bf Bayesian continuous space-time model of Burglaries}


\vspace{1.0cm}

Chaitanya Joshi$^1$ and Paul T. Brown $^1$ and Stephen Joe $^1$ and Dean Searle $^2$

\begin{description}

\item $^1 \;$ Department of Mathematics and Statistics, University of Waikato, Hamilton.

\item $^2 \;$ NZ Police, Waikato DHQ, Hamilton.

\end{description}

\end{flushleft}

%  ***** ADD ENOUGH VERTICAL SPACE HERE TO ENSURE THAT THE *****
%  ***** ABSTRACT (OR MAIN TEXT) STARTS 5 CM BELOW THE TOP *****

\vspace{0.75cm}

\noindent {\bf Abstract}. Building a predictive model of crime with good predictive accuracy has a great value in enabling efficient use of policing resources and reduction in crime. Building such models is not straightforward though due to the dynamic nature of the crime process. The crime not only evolves over both space and time, but is also related to several complex socio-economic factors, not all of which can be measured directly and accurately. The last decade or more has seen a surge in the effort to model crime more accurately. Many of the models developed so far have failed to capture the crime with a great degree of accuracy. The main reasons could be that all these models discretise the space using grid cells and that they are spatial, not spatio-temporal.
We fit a log Gaussian Cox process model using the INLA-SPDE approach. This not only allows us to capture crime as a process continuous in both space and time, but also allows us to include socio-economic factors as well as the 'near repeat' phenomenon. In this talk, we will discuss the model building process and the accuracy achieved.

\vskip 2mm

\noindent {\bf Keywords}.
Bayesian spatio-temporal model, INLA-SPDE, predicting crime

%\section{ First-level heading}
%The C98 head 1 style leaves a half-line spacing below a
%first-level heading. There should be one blank line above
%a first-level heading.
%        
%\subsection { Second-level heading}
%There should also be one blank line above a second- or
%third-level heading (but no extra space below them).
%
%Do not intent the first paragraph following a heading.
%Second and subsequent paragraphs are indented by one Tab
%character (= 3 mm). If footnotes are used, they should be
%placed at the foot of the page\footnote{ Footnotes are separated
%from the text by a blank line and a printed line of length 3.5 cm.
%They should be printed in 9-point Times Roman in single line spacing.}.
%        
%\subsubsection { Third-level heading}
%Please specify references using the conventions
%illustrated below. Each should begin on a new line, and
%second and subsequent lines should be on the same page
%indented by 3 mm.

%\subsection*{References}
%
%\begin{description}
%
%\item
%Barnett, J.A., Payne, R.W. and Yarrow, D. (1990).
%\textit{Yeasts: Characteristics and identification: Second Edition.}
%Cambridge: Cambridge University Press.
%
%\item
%(ed.) Barnett, V., Payne, R. and Steiner, R. (1995).
%\textit{Agricultural Sustainability: Economic, Environmental and
%Statistical Considerations}. Chichester: Wiley.
%
%\item
%Payne, R.W. (1997).
%\textit{Algorithm AS314 Inversion of matrices Statistics},
%\textbf{46}, 295--298.
%
%\item
%Payne, R.W. and Welham, S.J. (1990).
%A comparison of algorithms for combination of information in generally
%balanced designs.
%In: \textit{COMPSTAT90 Proceedings in Computational Statistics}, 297--302.
%Heidelberg: Physica-Verlag.
%
%\end{description}

\end{document}





