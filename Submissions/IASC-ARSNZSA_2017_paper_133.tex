\documentclass[12pt]{article}
% \documentstyle{iascars2017}

% \usepackage{iascars2017}

\pagestyle{myheadings}
\pagenumbering{arabic}
\topmargin 0pt \headheight 23pt \headsep 24.66pt
%\topmargin 11pt \headheight 12pt \headsep 13.66pt
\parindent = 3mm


\begin{document}


\begin{flushleft}


{\LARGE\bf Penalized Vector Generalized Additive Models}


\vspace{1.0cm}

T.~W.~Yee$^1$ and C.~Somchit$^2$
and C.~J.~Wild$^1$

\begin{description}

\item $^1 \;$ Department of Statistics,
University of Auckland,
New Zealand

\item $^2 \;$ Phayao University, Thailand

\end{description}

\end{flushleft}

%  ***** ADD ENOUGH VERTICAL SPACE HERE TO ENSURE THAT THE *****
%  ***** ABSTRACT (OR MAIN TEXT) STARTS 5 CM BELOW THE TOP *****

\vspace{0.75cm}

\noindent {\bf Abstract}.
Over the last two decades generalized additive
models (GAMs) have become an indispensible tool for modern data
analysis and regression. First-generation GAMs as developed by
Hastie and Tibshirani are based on backfitting (e.g., the \texttt{gam} R
package). Second-generation GAMs have automatic smoothing parameter
selection (e.g., the \texttt{mgcv} package by Simon Wood) and are based
on, e.g., P-splines. Until recently, these two implementations
were largely confined to the exponential family. However, since
the 1990s, the vector generalized linear and additive model
(VGLM/VGAM) classes were developed by Yee and coworkers, and these
are a much larger class of models. First-generation VGAMs were
based on vector splines and vector backfitting. This talk will
describe 2nd-generation VGAMs using O-splines and P-splines. We
illustrate them by examples, to show that automatic smoothing
parameter selection based on optimizing a predictive quantity such
as generalized cross validation can be very useful.  The speaker's
\texttt{VGAM} package implementation will be described.


\vskip 2mm

\noindent {\bf Keywords}.
Automatic smoothing parameter selection,
O-splines,
P-splines,
Vector generalized additive models,
VGAM R package



%\section{ First-level heading}
%The C98 head 1 style leaves a half-line spacing below a
%first-level heading. There should be one blank line above
%a first-level heading.
%
%\subsection { Second-level heading}
%There should also be one blank line above a second- or
%third-level heading (but no extra space below them).
%
%Do not intent the first paragraph following a heading.
%Second and subsequent paragraphs are indented by one Tab
%character (= 3 mm). If footnotes are used, they should be
%placed at the foot of the page\footnote{ Footnotes are separated
%from the text by a blank line and a printed line of length 3.5 cm.
%They should be printed in 9-point Times Roman in single line spacing.}.
%
%\subsubsection { Third-level heading}
%Please specify references using the conventions
%illustrated below. Each should begin on a new line, and
%second and subsequent lines should be on the same page
%indented by 3 mm.

\subsection*{References}

\begin{description}

\item
Yee, T.W. (2015).
\textit{Vector Generalized Linear and Additive Models: With
an Implementation in R}.
New York, USA: Springer.


\end{description}

\end{document}





