\documentclass[12pt]{article}
% \documentstyle{iascars2017}

% \usepackage{iascars2017}

\pagestyle{myheadings} 
\pagenumbering{arabic}
\topmargin 0pt \headheight 23pt \headsep 24.66pt
%\topmargin 11pt \headheight 12pt \headsep 13.66pt
\parindent = 3mm 


\begin{document}


\begin{flushleft}


{\LARGE\bf Scoring rules for prediction and classification challenges}


\vspace{1.0cm}

Matthew Parry$^1$

\begin{description}

\item $^1 \;$ Department of Mathematics \& Statistics, University of Otago, Dunedin, NZ


\end{description}

\end{flushleft}

%  ***** ADD ENOUGH VERTICAL SPACE HERE TO ENSURE THAT THE *****
%  ***** ABSTRACT (OR MAIN TEXT) STARTS 5 CM BELOW THE TOP *****

\vspace{0.75cm}

\noindent {\bf Abstract}. Prediction and classification challenges have become an exciting and useful feature of the statistical and machine learning community. For example, Good Judgement\textregistered\ Open asks forecasters to predict the probability of particular world events, and Kaggle.com regularly sets classification challenges. Challenge organizers typically publish a ranked list of the leading submissions and, ultimately, announce the winner of the challenge. However, in order for such a competition to be considered worth entering, the challenge organizers must be seen to evaluate the submissions in a fair and open manner. Scoring rules were devised precisely to solve this problem. Crucially, {\em proper} scoring rules elicit honest statements of belief about the outcome. If the challenge organizers use a proper scoring rule to evaluate submissions, a competitor's expected score under their true belief will be optimized by actually quoting that belief to the organizers. A proper scoring rule therefore rules out any possibility of a competitor gaming the challenge. We discuss a class of proper scoring rules called linear scoring rules that are specifically adapted to probabilistic binary classification. When applied in competition situations, we show that all linear scoring rules essentially balance the needs of organizers and competitors. We also develop scoring rules to score a sequence of predictions that are targeting a single outcome. These scoring rules discount predictions over time and appropriately weight prediction updates.

\vskip 2mm

\noindent {\bf Keywords}.
Probabilistic forecast, sequence, prequential principle, discounting


%\section{ First-level heading}
%The C98 head 1 style leaves a half-line spacing below a
%first-level heading. There should be one blank line above
%a first-level heading.
%        
%\subsection { Second-level heading}
%There should also be one blank line above a second- or
%third-level heading (but no extra space below them).
%
%Do not intent the first paragraph following a heading.
%Second and subsequent paragraphs are indented by one Tab
%character (= 3 mm). If footnotes are used, they should be
%placed at the foot of the page\footnote{ Footnotes are separated
%from the text by a blank line and a printed line of length 3.5 cm.
%They should be printed in 9-point Times Roman in single line spacing.}.
%        
%\subsubsection { Third-level heading}
%Please specify references using the conventions
%illustrated below. Each should begin on a new line, and
%second and subsequent lines should be on the same page
%indented by 3 mm.

\subsection*{References}

\begin{description}

\item Parry, M. (2016). \textit{Linear scoring rules for probabilistic binary classification}. Electronic Journal of Statistics, 10 (1), 1596--1607.




\end{description}

\end{document}





