\documentclass[12pt]{article}
% \documentstyle{iascars2017}

% \usepackage{iascars2017}

\pagestyle{myheadings} 
\pagenumbering{arabic}
\topmargin 0pt \headheight 23pt \headsep 24.66pt
%\topmargin 11pt \headheight 12pt \headsep 13.66pt
\parindent = 3mm 


\begin{document}


\begin{flushleft}


{\LARGE\bf Bayesign optimum   warranty length  under Type-II unified hybrid censoring scheme}


\vspace{1.0cm}

Tanmay Sen$^1$, Biswabrata Pradhan$^2$, Yogesh Mani Tripathi$^{1}$ and Ritwik Bhattacharya$^{3}$

\begin{description}

\item $^1 \;$ Department of Mathematics, Indian Institute of Technology, Patna 801106, India 
\item $^2 \;$ SQC and OR Unit, Indian Statistical Institute, 203 B.T. Road, Kolkata 700108, India 
\item $^3 \;$ Centro de Investigacionen Matematicas (CIMAT), Monterrey, Mexico 

\end{description}

\end{flushleft}

%  ***** ADD ENOUGH VERTICAL SPACE HERE TO ENSURE THAT THE *****
%  ***** ABSTRACT (OR MAIN TEXT) STARTS 5 CM BELOW THE TOP *****

\vspace{0.75cm}

\noindent {\bf Abstract}.This work considers determination of optimum warranty length under Type-II unified hybrid censoring scheme. Consumers are willing to purchase a highly reliable product with certain cost constraint. To assure the product reliability and also to remain profitable, the manufacturer provides warranties on product lifetime. Moreover, censoredlifetime data are available in practice, to assess the reliability of the product. Therefore, determination of an appropriate warranty length based on censored lifetime data is an important issue to the manufacturer.  It is assumed that the lifetime follows a lognormal distribution.  We consider a combine free replacement and pro-rata warranty policy (FRW/PRW). The life test is conducted under Type-II unified hybrid censoring scheme.  The warranty length is obtained by maximizing an expected utility function.The expectation is taken with respect to the posterior predictive model for time to failure given the available data obtained under Type-II unified hybrid censoring scheme. A real data set is analyzed to illustrate the proposed methodology.  We propose a non-linear prorate warranty policy and compare them with linear warranty policy.  It is observed that non-linear prorate warranty policy give larger warranty length with maximum profit

\vskip 2mm

\noindent {\bf Keywords}.
Lognormal distribution, FRW/PRW policies, Optimum warranty length, MH algorithm

\vfill
Email- sentanmay518@gmail.com, 
 Mob- 918084526317


%\section{ First-level heading}
%The C98 head 1 style leaves a half-line spacing below a
%first-level heading. There should be one blank line above
%a first-level heading.
%        
%\subsection { Second-level heading}
%There should also be one blank line above a second- or
%third-level heading (but no extra space below them).
%
%Do not intent the first paragraph following a heading.
%Second and subsequent paragraphs are indented by one Tab
%character (= 3 mm). If footnotes are used, they should be
%placed at the foot of the page\footnote{ Footnotes are separated
%from the text by a blank line and a printed line of length 3.5 cm.
%They should be printed in 9-point Times Roman in single line spacing.}.
%        
%\subsubsection { Third-level heading}
%Please specify references using the conventions
%illustrated below. Each should begin on a new line, and
%second and subsequent lines should be on the same page
%indented by 3 mm.

%\subsection*{References}

%\begin{description}

%\item
%Barnett, J.A., Payne, R.W. and Yarrow, D. (1990).
%\textit{Yeasts: Characteristics and identification: Second Edition.}
%Cambridge: Cambridge University Press.

%\item
%(ed.) Barnett, V., Payne, R. and Steiner, R. (1995).
%\textit{Agricultural Sustainability: Economic, Environmental and
%Statistical Considerations}. Chichester: Wiley.

%\item
%Payne, R.W. (1997).
%\textit{Algorithm AS314 Inversion of matrices Statistics},
%\textbf{46}, 295--298.

%\item
%Payne, R.W. and Welham, S.J. (1990).
%A comparison of algorithms for combination of information in generally
%balanced designs.
%In: \textit{COMPSTAT90 Proceedings in Computational Statistics}, 297--302.
%Heidelberg: Physica-Verlag.

%\end{description}

\end{document}





