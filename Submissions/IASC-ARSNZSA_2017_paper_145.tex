\documentclass[12pt]{article}
% \documentstyle{iascars2017}

% \usepackage{iascars2017}

\pagestyle{myheadings} 
\pagenumbering{arabic}
\topmargin 0pt \headheight 23pt \headsep 24.66pt
%\topmargin 11pt \headheight 12pt \headsep 13.66pt
\parindent = 3mm 


\begin{document}


\begin{flushleft}


{\LARGE\bf Model-based clustering for multivariate categorical data with
 dimension reduction}


\vspace{1.0cm}

Michio Yamamoto$^1$

\begin{description}

\item $^1 \;$ Graduate School of Environmental and Life Science, Okayama
						University,
Okayama, Japan

% \item $^2 \;$ Center for Applied Research in Computer Science,
% Applied Research Laboratory, Anyville, AB 12345, USA

\end{description}

\end{flushleft}

%  ***** ADD ENOUGH VERTICAL SPACE HERE TO ENSURE THAT THE *****
%  ***** ABSTRACT (OR MAIN TEXT) STARTS 5 CM BELOW THE TOP *****

\vspace{0.75cm}

\noindent {\bf Abstract}. A novel model-based clustering procedure for
multivariate categorical data is proposed. The proposed model assumes
that each response probability has a low-dimensional representation of
the cluster structure, which is constructed by weights for categorical
variables and scores for cluster representatives. For the visualization
of the cluster structure, we define low-dimensional scores for
individuals as convex combinations of scores for cluster
representatives, which may be interpretable in a similar manner to the
archetypal analysis developed by Cutler and Breiman (1994). Because the
proposed model has the so-called rotational indeterminacy, it is needed
to conduct rotation methods after parameter estimation to obtain
interpretable results. Instead of this two-step approach, we develop a
penalized likelihood procedure that imposes a sparsity-inducing penalty
on the weights for categorical variables. To optimize the proposed
penalized likelihood criterion, we develop an expectation-maximization
(EM) algorithm with gradient projection and coordinate descent. It is
shown that there is trade-off relation between the convergence rate of
the algorithm and the cluster recovery.

\vskip 2mm

\noindent {\bf Keywords}.  clustering, categorical data, dimension
reduction, EM algorithm, sparse estimation


%\section{ First-level heading}
%The C98 head 1 style leaves a half-line spacing below a
%first-level heading. There should be one blank line above
%a first-level heading.
%        
%\subsection { Second-level heading}
%There should also be one blank line above a second- or
%third-level heading (but no extra space below them).
%
%Do not intent the first paragraph following a heading.
%Second and subsequent paragraphs are indented by one Tab
%character (= 3 mm). If footnotes are used, they should be
%placed at the foot of the page\footnote{ Footnotes are separated
%from the text by a blank line and a printed line of length 3.5 cm.
%They should be printed in 9-point Times Roman in single line spacing.}.
%        
%\subsubsection { Third-level heading}
%Please specify references using the conventions
%illustrated below. Each should begin on a new line, and
%second and subsequent lines should be on the same page
%indented by 3 mm.

\subsection*{References}

\begin{description}

 \item Cutler, A., Breiman, L. (1994). Archetypal
						 analysis. \textit{Technometrics}, \textbf{36}, 338--347.

 \item Yamamoto, M., Hayashi, K. (2015). Clustering of multivariate
						 binary data with dimension reduction via
						 $L_{1}$-regularized likelihood
						 maximization. \textit{Pattern Recognition}, \textbf{48},
						 3959--3968.

% \item
% Barnett, J.A., Payne, R.W. and Yarrow, D. (1990).
% \textit{Yeasts: Characteristics and identification: Second Edition.}
% Cambridge: Cambridge University Press.

% \item
% (ed.) Barnett, V., Payne, R. and Steiner, R. (1995).
% \textit{Agricultural Sustainability: Economic, Environmental and
% Statistical Considerations}. Chichester: Wiley.

% \item
% Payne, R.W. (1997).
% \textit{Algorithm AS314 Inversion of matrices Statistics},
% \textbf{46}, 295--298.

% \item
% Payne, R.W. and Welham, S.J. (1990).
% A comparison of algorithms for combination of information in generally
% balanced designs.
% In: \textit{COMPSTAT90 Proceedings in Computational Statistics}, 297--302.
% Heidelberg: Physica-Verlag.

\end{description}

\end{document}





