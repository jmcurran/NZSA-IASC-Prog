\documentclass[12pt]{article}
% \documentstyle{iascars2017}

% \usepackage{iascars2017}

\pagestyle{myheadings} 
\pagenumbering{arabic}
\topmargin 0pt \headheight 23pt \headsep 24.66pt
%\topmargin 11pt \headheight 12pt \headsep 13.66pt
\parindent = 3mm 


\begin{document}


\begin{flushleft}


{\LARGE\bf  An information criterion for prediction with auxiliary variables under covariate shift}

\vspace{1.0cm}

Takahiro Ido$^1$, Shinpei Imori$^{12}$and Hidetoshi Shimodaira$^{23}$

\begin{description}

\item $^1 \;$ Graduate School of Engineering Science, Osaka University,
1-3 Machikaneyama, Toyonaka, Osaka, 560-8531, Japan

\item $^2 \;$ RIKEN Center for Advanced Intelligence Project (AIP), 
1-4-1 Nihonbashi, Chuo-ku, Tokyo, 103-0027, Japan

\item $^3 \;$ Graduate School of Informatics, Kyoto University,
Yoshida Honmachi, Sakyo-ku, Kyoto, 606-8501, Japan

\end{description}

\end{flushleft}

%  ***** ADD ENOUGH VERTICAL SPACE HERE TO ENSURE THAT THE *****
%  ***** ABSTRACT (OR MAIN TEXT) STARTS 5 CM BELOW THE TOP *****

\vspace{0.75cm}

\noindent {\bf Abstract}. 
It is beneficial for modeling data of interest to exploit secondary information. 
The secondary information is called auxiliary variables, which may not be observed in testing data because they are not of primary interest. 
In this paper, we incorporate the auxiliary variables into a framework of supervised learning. 
Furthermore, we consider a covariate shift situation that allows a density function of covariates to change between testing and training data. 
It is known that the Maximum Log-likelihood Estimate (MLE) is not a good estimator under model misspecification and the covariate shift. 
This problem can be resolved by the Maximum Weighted Log-likelihood Estimate (MWLE). 

When we have multiple candidate models, it needs to select the best candidate model where its optimality is measured by the expected Kullback-Leibler (KL) divergence. 
The Akaike information criterion (AIC) is a well known criterion based on the KL divergence and using the MLE.  
Therefore, its validity is not guaranteed when the MWLE is used under the covariate shift. 
An information criterion under the covariate shift was proposed in Shimodaira (2000, JSPI) but this criterion does not take use of the auxiliary variables into account. 
Hence, we resolve this problem by deriving a new criterion. 
In addition, simulations are conducted to examine the improvement. 

\vskip 2mm

\noindent {\bf Keywords}.
Auxiliary variables; Covariate shift; Information criterion; Kullback-Leibler divergence; Misspecification; Predictions. 


%\section{ First-level heading}
%The C98 head 1 style leaves a half-line spacing below a
%first-level heading. There should be one blank line above
%a first-level heading.
%        
%\subsection { Second-level heading}
%There should also be one blank line above a second- or
%third-level heading (but no extra space below them).
%
%Do not intent the first paragraph following a heading.
%Second and subsequent paragraphs are indented by one Tab
%character (= 3 mm). If footnotes are used, they should be
%placed at the foot of the page\footnote{ Footnotes are separated
%from the text by a blank line and a printed line of length 3.5 cm.
%They should be printed in 9-point Times Roman in single line spacing.}.
%        
%\subsubsection { Third-level heading}
%Please specify references using the conventions
%illustrated below. Each should begin on a new line, and
%second and subsequent lines should be on the same page
%indented by 3 mm.

%\subsection*{References}

%\begin{description}

%\item
%Barnett, J.A., Payne, R.W. and Yarrow, D. (1990).
%\textit{Yeasts: Characteristics and identification: Second Edition.}
%Cambridge: Cambridge University Press.

%\item
%(ed.) Barnett, V., Payne, R. and Steiner, R. (1995).
%\textit{Agricultural Sustainability: Economic, Environmental and
%Statistical Considerations}. Chichester: Wiley.

%\item
%Payne, R.W. (1997).
%\textit{Algorithm AS314 Inversion of matrices Statistics},
%\textbf{46}, 295--298.

%\item
%Payne, R.W. and Welham, S.J. (1990).
%A comparison of algorithms for combination of information in generally
%balanced designs.
%In: \textit{COMPSTAT90 Proceedings in Computational Statistics}, 297--302.
%Heidelberg: Physica-Verlag.

%\end{description}

\end{document}





