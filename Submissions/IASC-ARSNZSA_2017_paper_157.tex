\documentclass[12pt]{article}
% \documentstyle{iascars2017}

% \usepackage{iascars2017}

\pagestyle{myheadings} 
\pagenumbering{arabic}
\topmargin 0pt \headheight 23pt \headsep 24.66pt
%\topmargin 11pt \headheight 12pt \headsep 13.66pt
\parindent = 3mm 


\begin{document}


\begin{flushleft}


{\LARGE\bf An Advanced Approach for Time Series Forecasting using Deep Learning}


\vspace{1.0cm}

Balaram Panda$^1$

\begin{description}

\item $^1 \;$ Data Scientist, Inland Revenue Department, New Zealand

\end{description}

\end{flushleft}

%  ***** ADD ENOUGH VERTICAL SPACE HERE TO ENSURE THAT THE *****
%  ***** ABSTRACT (OR MAIN TEXT) STARTS 5 CM BELOW THE TOP *****

\vspace{0.75cm}

\noindent {\bf Abstract}. Time series forecasting is a decade-long research and which is being evolving day by day. Due to the recent advancement is deep learning technique many of the complex problems have been solved using deep learning. Deep learning techniques have shown tremendous better performance in supervised learning problem. One of the reasons for this success is the ability of deep feedforward network methods to learn multiple feature interaction for a single instance. However, the time-dependent nature not being captured by deep feedforward network till the evolution of RNN(recurrent neural network) and LSTM(long short term memory) network architecture. This paper reveals the success of LSTM time series in comparison with ARIMA and other standard approaches for time series modeling. A sensitivity analysis is also conducted to explore the effect of hyper parameter tuning on LSTM model to reduce the time series forecasting error.  We also derive practical advice from our empirical results for those interested in getting most out of LSTM time series for modern time series forecasting.

\vskip 2mm

\noindent {\bf Keywords}.
Deep Learning, Time Series, LSTM, RNN


%\section{ First-level heading}
%The C98 head 1 style leaves a half-line spacing below a
%first-level heading. There should be one blank line above
%a first-level heading.
%        
%\subsection { Second-level heading}
%There should also be one blank line above a second- or
%third-level heading (but no extra space below them).
%
%Do not intent the first paragraph following a heading.
%Second and subsequent paragraphs are indented by one Tab
%character (= 3 mm). If footnotes are used, they should be
%placed at the foot of the page\footnote{ Footnotes are separated
%from the text by a blank line and a printed line of length 3.5 cm.
%They should be printed in 9-point Times Roman in single line spacing.}.
%        
%\subsubsection { Third-level heading}
%Please specify references using the conventions
%illustrated below. Each should begin on a new line, and
%second and subsequent lines should be on the same page
%indented by 3 mm.

\subsection*{References}

\begin{description}

\item
Längkvist, Martin, Lars Karlsson, and Amy Loutfi. "A review of unsupervised feature learning and deep learning for time-series modeling." Pattern Recognition Letters 42 (2014): 11-24.

\item
Zheng, Yi, et al. "Time series classification using multi-channels deep convolutional neural networks." International Conference on Web-Age Information Management. Springer, Cham, 2014.

\end{description}

\end{document}





