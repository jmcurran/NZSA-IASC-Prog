\documentclass[12pt]{article}
% \documentstyle{iascars2017}

% \usepackage{iascars2017}

\pagestyle{myheadings}
\pagenumbering{arabic}
\topmargin 0pt \headheight 23pt \headsep 24.66pt
%\topmargin 11pt \headheight 12pt \headsep 13.66pt
\parindent = 3mm


\begin{document}


%\begin{flushleft}


{\LARGE\bf An EWMA Chart for Monitoring Covariance Matrix Based on Dissimilarity Index}


\vspace{1.0cm}

Longcheen Huwang

\begin{description}

\item Institute of Statistics,
National Tsing Hua University,
Hsinchu 30043,\\ Taiwan

\end{description}

%\end{flushleft}

%  ***** ADD ENOUGH VERTICAL SPACE HERE TO ENSURE THAT THE *****
%  ***** ABSTRACT (OR MAIN TEXT) STARTS 5 CM BELOW THE TOP *****

\vspace{0.75cm}

\noindent {\bf Abstract}. In this talk, we propose an EWMA chart for monitoring covariance matrix based on the dissimilarity index of two matrices. It is different from the conventional EWMA charts for monitoring covariance matrix which are either based on comparing the sum or product or both of the eigenvalues of the estimated EWMA covariance matrix with those of the IC covariance matrix. The proposed chart essentially monitors covariance matrix by comparing the individual eigenvalues of the estimated EWMA covariance matrix with those of the estimated covariance matrix from the IC phase I data.  We evaluate the performance of the proposed chart by comparing it with the best existing chart under the multivariate normal process. Furthermore, to prevent the control limit of the proposed EMMA chart using the limited IC phase I data from having extensively excessive false alarms, we use a bootstrap method to adjust the control limit to guarantee that the proposed chart has the actual IC average run length not less than the nominal one with a certain probability. Finally, we use an example to demonstrate the applicability and implementation of the proposed chart. 

\vskip 2mm

\noindent {\bf Keywords}. Average run length, dissimilarity index, EWMA; out-of-control



%\section{ First-level heading}
%The C98 head 1 style leaves a half-line spacing below a
%first-level heading. There should be one blank line above
%a first-level heading.
%
%\subsection { Second-level heading}
%There should also be one blank line above a second- or
%third-level heading (but no extra space below them).
%
%Do not intent the first paragraph following a heading.
%Second and subsequent paragraphs are indented by one Tab
%character (= 3 mm). If footnotes are used, they should be
%placed at the foot of the page\footnote{ Footnotes are separated
%from the text by a blank line and a printed line of length 3.5 cm.
%They should be printed in 9-point Times Roman in single line spacing.}.
%
%\subsubsection { Third-level heading}
%Please specify references using the conventions
%illustrated below. Each should begin on a new line, and
%second and subsequent lines should be on the same page
%indented by 3 mm.

\subsection*{References}

\begin{description}

\item
Hawkins, D.M. and Maboudou-Tchao E.M. (2008). Multivariate exponentially weighted moving covariance matrix. {\em Technometrics}, {\bf 50}, 155-166.
\item
Kano, M., Hasebe, S. and Hashimoto, I. (2002). Statistical process monitoring based on dissimilarity of process data. {\em AIChE Journal}, {\bf 48}, 1231-1240.


\end{description}

\end{document}
