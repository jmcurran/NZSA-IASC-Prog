\documentclass[12pt]{article}
% \documentstyle{iascars2017}

% \usepackage{iascars2017}

\pagestyle{myheadings} 
\pagenumbering{arabic}
\textheight 680pt
\parindent = 3mm 
 


\begin{document}


\begin{flushleft}


{\LARGE\bf Tolerance limits for the reliability of semiconductor devices using longitudinal data}


\vspace{1.0cm}

Vera Hofer$^1$, Johannes Leitner$^1$, Horst Lewitschnig$^2$ and Thomas Nowak$^1$

\begin{description}

\item $^1 \;$ Institute of Statistics and Operations Research, University of Graz, Graz, Austria
\item $^2 \;$ Infineon Technologies Austria AG, Villach, Austria

\end{description}

\end{flushleft}

\vspace{0.75cm}

\noindent {\bf Abstract}. 

Especially in the automotive industry, semiconductor devices are key components for the proper functioning of the entire vehicle. Therefore, issues concerning the reliability of these components are of crucial importance to manufacturers of semiconductor devices.

In this quality control task, we consider longitudinal data from high temperature operating life tests. Manufacturers then need to find appropriate tolerance limits for their final electrical product tests, such that the proper functioning of their devices is ensured. Based on these datasets, we compute tolerance limits that could then be used by automated test equipment for the ongoing quality control process. Devices with electrical parameters within their respective tolerance limits can successfully finish the production line, while all other devices will be discarded. In calculating these tolerance limits, our approach consists of two steps: First, the observed measurements are transformed in order to capture measurement biases and gauge repeatability and reproducibility. Then, in the second step, we compute tolerance limits based on a multivariate copula model with skew normal distributed margins. In order to solve the resulting optimization problem, we propose a new derivative-free optimization procedure.

The capability of the model is demonstrated by computing optimal tolerance limits for several drift patterns that are expected to cover a wide range of scenarios. Based on these computations, we show the resulting yield losses and analyze the performance of the tolerance limits a large simulation study.

\noindent {\bf Acknowledgment}

This work was supported by the ECSEL Joint Undertaking under grant agreement No. 662133 - PowerBase. This Joint Undertaking receives support from the European Union's Horizon 2020 research and innovation programme and Austria, Belgium, Germany, Italy, Netherlands, Norway, Slovakia, Spain and United Kingdom.

\vskip 2mm

\noindent {\bf Keywords}. quality control, tolerance limits, copulas, skew normal distribution


\end{document}





