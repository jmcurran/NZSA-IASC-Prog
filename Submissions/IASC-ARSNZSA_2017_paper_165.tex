\documentclass[12pt]{article}
% \documentstyle{iascars2017}

% \usepackage{iascars2017}

\pagestyle{myheadings} 
\pagenumbering{arabic}
\topmargin 0pt \headheight 23pt \headsep 24.66pt
%\topmargin 11pt \headheight 12pt \headsep 13.66pt
\parindent = 3mm 


\begin{document}


\begin{flushleft}


{\LARGE\bf Statistical generalized  derivative applied to the profile likelihood estimation in a mixture of semiparametric models}


\vspace{1.0cm}

Yuichi Hirose$^1$ and Ivy Liu$^2$

\begin{description}

\item $^1 \;$ School of Mathematics and Statistics,
Victoria University of Wellington

\item $^2 \;$ School of Mathematics and Statistics,
Victoria University of Wellington

\end{description}

\end{flushleft}

%  ***** ADD ENOUGH VERTICAL SPACE HERE TO ENSURE THAT THE *****
%  ***** ABSTRACT (OR MAIN TEXT) STARTS 5 CM BELOW THE TOP *****

\vspace{0.75cm}

\noindent {\bf Abstract}. There is a difficulty in finding an estimate of variance of the profile likelihood estimator in the joint model of longitudinal and survival data. 
We solve the difficulty by introducing the ``statistical generalized  derivative''.
The derivative is used to show the asymptotic normality of the estimator without assuming the second derivative of the density function in the model exists.


\vskip 2mm

\noindent {\bf Keywords}.
Efficiency,
Efficient information bound,
Efficient score,
Implicitly defined function,
Profile likelihood,
Semi-parametric model,
Joint model,
EM algorithm,
Mixture model


%\section{ First-level heading}
%The C98 head 1 style leaves a half-line spacing below a
%first-level heading. There should be one blank line above
%a first-level heading.
%        
%\subsection { Second-level heading}
%There should also be one blank line above a second- or
%third-level heading (but no extra space below them).
%
%Do not intent the first paragraph following a heading.
%Second and subsequent paragraphs are indented by one Tab
%character (= 3 mm). If footnotes are used, they should be
%placed at the foot of the page\footnote{ Footnotes are separated
%from the text by a blank line and a printed line of length 3.5 cm.
%They should be printed in 9-point Times Roman in single line spacing.}.
%        
%\subsubsection { Third-level heading}
%Please specify references using the conventions
%illustrated below. Each should begin on a new line, and
%second and subsequent lines should be on the same page
%indented by 3 mm.

\subsection*{References}

\begin{description}

\item Hsieh, F.,  Tseng, Y.K. and Wang, J.L. (2006). 
\textit{Joint modeling of survival and longitudinal data: likelihood approach revisited.}
Biometrics {\bf 62}, 1037--1043. 

\item Hirose, Y. (2016).  
\textit{On differentiability of implicitly defined
function in semi-parametric profile
likelihood estimation.}
Bernoulli \textbf{22} 589--614.


\item \textsc{Preedalikit et al.} (2016). 
\textit{Joint modeling of survival and longitudinal ordered data using a semiparametric approach.}
Australian \& New Zealand Journal of Statistics  {\bf 58}, 153--172. 



\end{description}

\end{document}





