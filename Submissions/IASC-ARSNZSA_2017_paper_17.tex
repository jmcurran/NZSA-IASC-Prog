\documentclass[12pt]{article}
% \documentstyle{iascars2017}

% \usepackage{iascars2017}

\pagestyle{myheadings}
\pagenumbering{arabic}
\topmargin 0pt \headheight 23pt \headsep 24.66pt
%\topmargin 11pt \headheight 12pt \headsep 13.66pt
\parindent = 3mm


\begin{document}


\begin{flushleft}


{\LARGE\bf Analysis of Multivariate Binary Longitudinal Data: Metabolic Syndrome During Menopausal Transition}


\vspace{1.0cm}

Geoff Jones$^1$

\begin{description}

\item $^1 \;$ Institute of Fundamental Sciences, Massey University,
Palmerston North, NZ

\end{description}

\end{flushleft}

%  ***** ADD ENOUGH VERTICAL SPACE HERE TO ENSURE THAT THE *****
%  ***** ABSTRACT (OR MAIN TEXT) STARTS 5 CM BELOW THE TOP *****

\vspace{0.75cm}

\noindent {\bf Abstract}. Metabolic syndrome (MetS) is a major multifactorial condition that predisposes adults to type 2 diabetes and cardiovascular disease. It is defined as having at least three of five cardiometabolic risk components:  1) high fasting triglyceride level, 2) low high-density lipoprotein (HDL) cholesterol, 3) elevated fasting plasma glucose, 4) large waist circumference (abdominal obesity) and 5) hypertension. In the US Study of Women’s Health Across the Nation (SWAN), a 15-year multi-centre prospective cohort study of women from five racial/ethnic groups, the incidence of MetS increased as midlife women underwent the menopausal transition (MT). A model is sought to examine the interdependent progression of the five MetS components and the influence of demographic covariates.

\vskip 2mm

\noindent {\bf Keywords}.
Multivariate binary data, longitudinal analysis, metabolic syndrome


%\section{ First-level heading}
%The C98 head 1 style leaves a half-line spacing below a
%first-level heading. There should be one blank line above
%a first-level heading.
%
%\subsection { Second-level heading}
%There should also be one blank line above a second- or
%third-level heading (but no extra space below them).
%
%Do not intent the first paragraph following a heading.
%Second and subsequent paragraphs are indented by one Tab
%character (= 3 mm). If footnotes are used, they should be
%placed at the foot of the page\footnote{ Footnotes are separated
%from the text by a blank line and a printed line of length 3.5 cm.
%They should be printed in 9-point Times Roman in single line spacing.}.
%
%\subsubsection { Third-level heading}
%Please specify references using the conventions
%illustrated below. Each should begin on a new line, and
%second and subsequent lines should be on the same page
%indented by 3 mm.

%\subsection*{References}
%
%\begin{description}
%
%\item
%Barnett, J.A., Payne, R.W. and Yarrow, D. (1990).
%\textit{Yeasts: Characteristics and identification: Second Edition.}
%Cambridge: Cambridge University Press.
%
%\item
%(ed.) Barnett, V., Payne, R. and Steiner, R. (1995).
%\textit{Agricultural Sustainability: Economic, Environmental and
%Statistical Considerations}. Chichester: Wiley.
%
%\item
%Payne, R.W. (1997).
%\textit{Algorithm AS314 Inversion of matrices Statistics},
%\textbf{46}, 295--298.
%
%\item
%Payne, R.W. and Welham, S.J. (1990).
%A comparison of algorithms for combination of information in generally
%balanced designs.
%In: \textit{COMPSTAT90 Proceedings in Computational Statistics}, 297--302.
%Heidelberg: Physica-Verlag.
%
%\end{description}

\end{document}





