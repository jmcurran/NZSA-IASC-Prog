\documentclass[12pt]{article}
% \documentstyle{iascars2017}

% \usepackage{iascars2017}

\pagestyle{myheadings} 
\pagenumbering{arabic}
\topmargin 0pt \headheight 23pt \headsep 24.66pt
%\topmargin 11pt \headheight 12pt \headsep 13.66pt
\parindent = 3mm 


\begin{document}


\begin{flushleft}


{\LARGE\bf Hierarchical Structural Component Analysis of Gene-Environment Interactions}


\vspace{1.0cm}

Sungkyoung Choi$^1$ , Seungyeoun Lee$^2$, and Taesung Park$^3$

\begin{description}

\item $^1 \;$ 1Department of Pharmacology, Yonsei University College of Medicine, Seoul, Korea

\item $^2 \;$ Department of Mathematics and Statistics, Sejong University,
Seoul, Korea

\item $^3 \;$ Department of  Statistics, Seoul National University,
Seoul, Korea

\end{description}

\end{flushleft}

%  ***** ADD ENOUGH VERTICAL SPACE HERE TO ENSURE THAT THE *****
%  ***** ABSTRACT (OR MAIN TEXT) STARTS 5 CM BELOW THE TOP *****

\vspace{0.75cm}

\noindent {\bf Abstract}. Gene-environment interactions (GEI) are known to be one possible avenue for addressing the missing heritability problem in genome-wide association studies (GWAS). Although many statistical methods have been proposed for identifying and analyzing GEI, most of these consider interactions between a single genetic variants such as single nucleotide polymorphism (SNPs) by the environment. In this study, we proposed a new statistical method for gene-based GEI analysis, Hierarchical structural CoMponent analysis of Gene-Environment Interaction (HisCoM-GEI). HisCoM-GEI is based on generalized structured component analysis, and can consider hierarchical structural relationships among SNPs in a gene. HisCoM-GEI can effectively aggregate all possible pairwise SNP-Environment interactions into a latent variable by imposing a ridge penalty, from which it then performs GEI analysis. Furthermore, HisCoM-GEI can evaluate both gene-level and SNP-level analyses. We applied the HisCoM-GEI to the cohort data of the Korea Associated Resource (KARE) consortium to identify GEIs between genes and alcohol intake on the blood pressure traits.

\vskip 2mm

\noindent {\bf Keywords}.
Gene-environment interaction, SNP, gene, GWAS


%\section{ First-level heading}
%The C98 head 1 style leaves a half-line spacing below a
%first-level heading. There should be one blank line above
%a first-level heading.
%        
%\subsection { Second-level heading}
%There should also be one blank line above a second- or
%third-level heading (but no extra space below them).
%
%Do not intent the first paragraph following a heading.
%Second and subsequent paragraphs are indented by one Tab
%character (= 3 mm). If footnotes are used, they should be
%placed at the foot of the page\footnote{ Footnotes are separated
%from the text by a blank line and a printed line of length 3.5 cm.
%They should be printed in 9-point Times Roman in single line spacing.}.
%        
%\subsubsection { Third-level heading}
%Please specify references using the conventions
%illustrated below. Each should begin on a new line, and
%second and subsequent lines should be on the same page
%indented by 3 mm.


\end{document}





