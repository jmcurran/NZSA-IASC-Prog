\documentclass[12pt]{article}
% \documentstyle{iascars2017}

% \usepackage{iascars2017}

\pagestyle{myheadings} 
\pagenumbering{arabic}
\topmargin 0pt \headheight 23pt \headsep 24.66pt
%\topmargin 11pt \headheight 12pt \headsep 13.66pt
\parindent = 3mm 


\begin{document}


\begin{flushleft}


{\LARGE\bf Testing for genetic associations in arbitrarily structured populations}


\vspace{1.0cm}

Minsun Song$^1$

\begin{description}

\item $^1 \;$ Department of Statistics, Sookmyung Women's University,
Seoul, 04310, Korea 

\end{description}

\end{flushleft}

%  ***** ADD ENOUGH VERTICAL SPACE HERE TO ENSURE THAT THE *****
%  ***** ABSTRACT (OR MAIN TEXT) STARTS 5 CM BELOW THE TOP *****

\vspace{0.75cm}

\noindent {\bf Abstract}.We present a new statistical test of association between a trait and genetic markers, which we theoretically and practically prove to be robust to arbitrarily complex population structure. The statistical test involves a set of parameters that can be directly estimated from large-scale genotyping data, such as those measured in genome-wide associations studies. We also derive a new set of methodologies, called a genotype-conditional association test, shown to provide accurate association tests in populations with complex structures, manifested in both the genetic and non-genetic contributions to the trait. We demonstrate the proposed method on a large simulation study and on the real data. Our proposed framework provides a substantially different approach to the problem from existing methods.


\vskip 2mm

\noindent {\bf Keywords}.
Genome-wide association studies,  Latent variable, Population structure


%\section{ First-level heading}
%The C98 head 1 style leaves a half-line spacing below a
%first-level heading. There should be one blank line above
%a first-level heading.
%        
%\subsection { Second-level heading}
%There should also be one blank line above a second- or
%third-level heading (but no extra space below them).
%
%Do not intent the first paragraph following a heading.
%Second and subsequent paragraphs are indented by one Tab
%character (= 3 mm). If footnotes are used, they should be
%placed at the foot of the page\footnote{ Footnotes are separated
%from the text by a blank line and a printed line of length 3.5 cm.
%They should be printed in 9-point Times Roman in single line spacing.}.
%        
%\subsubsection { Third-level heading}
%Please specify references using the conventions
%illustrated below. Each should begin on a new line, and
%second and subsequent lines should be on the same page
%indented by 3 mm.




\end{document}






