\documentclass[12pt]{article}
% \documentstyle{iascars2017}

% \usepackage{iascars2017}

\pagestyle{myheadings} 
\pagenumbering{arabic}
\topmargin 0pt \headheight 23pt \headsep 24.66pt
%\topmargin 11pt \headheight 12pt \headsep 13.66pt
\parindent = 3mm 


\begin{document}


\begin{flushleft}


{\LARGE\bf Prior-based Bayesian Information Criterion }


\vspace{1.0cm}
M.~J. Bayarri$^1$, James O.  Berger$^2$, Woncheol Jang$^3$, Surajit Ray$^4$, Luis R. Pericchi$^5$ and Ingmar Visser$^6$

\begin{description}

\item $^1 \;$ University of Valencia, Valencia, Spain
\item $^2 \;$ Duke University, Durham NC, USA
\item $^3 \;$ Seoul National University, Seoul, Korea
\item $^4 \;$ University of Glasgow, Glasgow, UK
\item $^5 \;$ University of Puerto Rico, San Juan, Puerto Rico
\item $^6 \;$ University of Amsterdam, Amsterdam, The Netherlands

 
\end{description}

\end{flushleft}

%  ***** ADD ENOUGH VERTICAL SPACE HERE TO ENSURE THAT THE *****
%  ***** ABSTRACT (OR MAIN TEXT) STARTS 5 CM BELOW THE TOP *****

\vspace{0.75cm}

\noindent {\bf Abstract}. We present a new approach to model selection and Bayes factor determination, based on Laplace expansions (as in BIC), which we call
Prior-based Bayes Information Criterion (PBIC).
In this approach, the Laplace expansion is only done with the likelihood function, and then
a suitable prior distribution is chosen to allow exact computation of the
(approximate) marginal likelihood arising from the Laplace approximation and the prior.
The result is a closed-form expression similar to BIC, but now involves a term arising from the
prior distribution (which BIC ignores) and also incorporates the idea that different
parameters can have different effective sample sizes (whereas BIC only allows one overall
sample size $n$). We
also consider a modification of PBIC which is more favorable to complex models.

 

\vskip 2mm

\noindent {\bf Keywords}.
Bayes factors, model selection, Cauchy priors, consistency,
 effective sample size, Fisher information,
 Laplace expansions, robust priors
 


%\section{ First-level heading}
%The C98 head 1 style leaves a half-line spacing below a
%first-level heading. There should be one blank line above
%a first-level heading.
%        
%\subsection { Second-level heading}
%There should also be one blank line above a second- or
%third-level heading (but no extra space below them).
%
%Do not intent the first paragraph following a heading.
%Second and subsequent paragraphs are indented by one Tab
%character (= 3 mm). If footnotes are used, they should be
%placed at the foot of the page\footnote{ Footnotes are separated
%from the text by a blank line and a printed line of length 3.5 cm.
%They should be printed in 9-point Times Roman in single line spacing.}.
%        
%\subsubsection { Third-level heading}
%Please specify references using the conventions
%illustrated below. Each should begin on a new line, and
%second and subsequent lines should be on the same page
%indented by 3 mm.

\end{document}

\subsection*{References}

\begin{description}

\item
Barnett, J.A., Payne, R.W. and Yarrow, D. (1990).
\textit{Yeasts: Characteristics and identification: Second Edition.}
Cambridge: Cambridge University Press.

\item
(ed.) Barnett, V., Payne, R. and Steiner, R. (1995).
\textit{Agricultural Sustainability: Economic, Environmental and
Statistical Considerations}. Chichester: Wiley.

\item
Payne, R.W. (1997).
\textit{Algorithm AS314 Inversion of matrices Statistics},
\textbf{46}, 295--298.

\item
Payne, R.W. and Welham, S.J. (1990).
A comparison of algorithms for combination of information in generally
balanced designs.
In: \textit{COMPSTAT90 Proceedings in Computational Statistics}, 297--302.
Heidelberg: Physica-Verlag.

\end{description}

\end{document}





