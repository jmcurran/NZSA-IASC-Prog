\documentclass[12pt]{article}
% \documentstyle{iascars2017}

% \usepackage{iascars2017}

\pagestyle{myheadings} 
\pagenumbering{arabic}
\topmargin 0pt \headheight 23pt \headsep 24.66pt
%\topmargin 11pt \headheight 12pt \headsep 13.66pt
\parindent = 3mm 


\begin{document}


\begin{flushleft}


{\LARGE\bf Bayesian semiparametric hierarchical models for longitudinal data analysis with application to dose-response studies}


\vspace{1.0cm}

Taeryon Choi

\begin{description}

\item Department of Statistics, Korea University, South Korea

\end{description}

\end{flushleft}

%  ***** ADD ENOUGH VERTICAL SPACE HERE TO ENSURE THAT THE *****
%  ***** ABSTRACT (OR MAIN TEXT) STARTS 5 CM BELOW THE TOP *****

\vspace{0.75cm}

\noindent {\bf Abstract}.
In this work, we propose semiparametric Bayesian hierarchical additive mixed effects models for analyzing either longitudinal data or clustered data with applications to dose-response studies. In the semiparametric mixed effects model structure, we estimate nonparametric smoothing functions of continuous covariates by using a spectral representation of Gaussian processes and the subject-specific random effects by using Dirichlet process mixtures. In this framework, we develop semiparametric mixed effects models that include normal regression and quantile regressions with or without shape restrictions. In addition, we deal with the Bayesian nonparametric measurement error models, or errors-in-variable regression models, using Fourier series and Dirchlet process mixtures, in which the true covariate is not observable, but the surrogate of the true covariate, is only observed. The proposed methodology is compared with other existing approaches to additive mixed models in simulation studies and benchmark data examples. More importantly, we consider a real data application for dose-response analysis, in which measurement errors and shape constraints in the regression functions need to be incorporated with inter-study variability. 

\vskip 2mm

\noindent {\bf Keywords}.
Cadmium toxicity, Cosine series, Dose-response study, Hierarchical Model, Measurement errors, Shape restriction


%\section{ First-level heading}
%The C98 head 1 style leaves a half-line spacing below a
%first-level heading. There should be one blank line above
%a first-level heading.
%        
%\subsection { Second-level heading}
%There should also be one blank line above a second- or
%third-level heading (but no extra space below them).
%
%Do not intent the first paragraph following a heading.
%Second and subsequent paragraphs are indented by one Tab
%character (= 3 mm). If footnotes are used, they should be
%placed at the foot of the page\footnote{ Footnotes are separated
%from the text by a blank line and a printed line of length 3.5 cm.
%They should be printed in 9-point Times Roman in single line spacing.}.
%        
%\subsubsection { Third-level heading}
%Please specify references using the conventions
%illustrated below. Each should begin on a new line, and
%second and subsequent lines should be on the same page
%indented by 3 mm.

\end{document}


