\documentclass[12pt]{article}
% \documentstyle{iascars2017}

% \usepackage{iascars2017}

\pagestyle{myheadings} 
\pagenumbering{arabic}
\topmargin 0pt \headheight 23pt \headsep 24.66pt
%\topmargin 11pt \headheight 12pt \headsep 13.66pt
\parindent = 3mm 


\begin{document}


\begin{flushleft}


{\LARGE\bf Wavelet-based Power Transformation of non-Gaussian Long Memory Time Series}


\vspace{1.0cm}

Kyungduk Ko$^1$ and Chul Eung Kim$^2$

\begin{description}

\item $^1 \;$ Department of Mathematics, Boise State University, Boise, ID 83725, USA

\item $^2 \;$ Department of Applied Statistics, Yeonsei University, Seoul, Korea

\end{description}

\end{flushleft}

%  ***** ADD ENOUGH VERTICAL SPACE HERE TO ENSURE THAT THE *****
%  ***** ABSTRACT (OR MAIN TEXT) STARTS 5 CM BELOW THE TOP *****

\vspace{0.75cm}

\noindent {\bf Abstract}. We consider a power transformation through the well-known Box-cox transformation to induce normality from non-Gaussian long memory processes and propose a Bayesian method to simultaneously  estimate the transformation parameter and long memory parameter. To ease computational burdens due to the dense variance-covariance matrix of long memory time series, we base our statistical inference on the wavelet domain rather than the original data domain. For a joint estimation of the parameters of interest, posterior estimations are carried out via Markov chain Monte Carlo (MCMC). An application to German stock return data is presented.  
\vskip 2mm

\noindent {\bf Keywords}.
Box-Cox transformation, Discrete wavelet transform, Long memory, MCMC, Normality
%\section{ First-level heading}
%The C98 head 1 style leaves a half-line spacing below a
%first-level heading. There should be one blank line above
%a first-level heading.
%        
%\subsection { Second-level heading}
%There should also be one blank line above a second- or
%third-level heading (but no extra space below them).
%
%Do not intent the first paragraph following a heading.
%Second and subsequent paragraphs are indented by one Tab
%character (= 3 mm). If footnotes are used, they should be
%placed at the foot of the page\footnote{ Footnotes are separated
%from the text by a blank line and a printed line of length 3.5 cm.
%They should be printed in 9-point Times Roman in single line spacing.}.
%        
%\subsubsection { Third-level heading}
%Please specify references using the conventions
%illustrated below. Each should begin on a new line, and
%second and subsequent lines should be on the same page
%indented by 3 mm.

\subsection*{References}

\begin{description}

\item
Dahlhaus, R. (1990). Efficient location and regression estimation for long range dependent regression models. 
\textit{Annuals of Statistics}, 23, 1029--1047.

\item
Ko, K. and Lee, J. (2008). Confidence intervals for long memory regressions. \textit{Statistics and Probability Letters}, 78, 1894--1902.

\item
Lee, J. and Ko, K. (2007). One-way analysis of variance with long memory errors and its application to stock return data.
\textit{Applied Stochastic Models in Business and Industry},
\textbf{23}, 493--502.

\end{description}

\end{document}





