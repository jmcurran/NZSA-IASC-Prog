\documentclass[12pt]{article}
% \documentstyle{iascars2017}

% \usepackage{iascars2017}

\pagestyle{myheadings} 
\pagenumbering{arabic}
\topmargin 0pt \headheight 23pt \headsep 24.66pt
%\topmargin 11pt \headheight 12pt \headsep 13.66pt
\parindent = 3mm 


\begin{document}


\begin{flushleft}


{\LARGE\bf Meta-analytic Principal Component Analysis in Integrative Omics Application}


\vspace{1.0cm}

SungHwan Kim$^1$ and George C. Tseng$^2$

\begin{description}

\item $^1 \;$ Department of Statistics, Keimyung Univerisity, Deagu 42601, South Korea

\item $^2 \;$ Department of Biostatistics, University of Pittsburgh, Pittsburgh 15213, USA

\end{description}

\end{flushleft}

%  ***** ADD ENOUGH VERTICAL SPACE HERE TO ENSURE THAT THE *****
%  ***** ABSTRACT (OR MAIN TEXT) STARTS 5 CM BELOW THE TOP *****

\vspace{0.75cm}

\noindent {\bf Abstract}. With the prevalent usage of microarray and massively parallel sequencing, numerous high-throughput omics datasets have become available in the public domain. Integrating abundant information among omics datasets is critical to elucidate biological mechanisms. Due to the high- dimensional nature of the data, methods such as principal component analysis (PCA) have been widely applied, aiming at effective dimension reduction and exploratory visualization. In this paper, we combine multiple omics datasets of identical or similar biological hypothesis and introduce two variations of meta-analytic framework of PCA, namely MetaPCA. Regularization is further incorporated to facilitate sparse feature selection in MetaPCA. We apply MetaPCA and sparse MetaPCA to simulations, three transcriptomic meta-analysis studies in yeast cell cycle, prostate cancer, mouse metabolism, and a TCGA pan-cancer methylation study. The result shows improved accuracy, robustness and exploratory visualization of the proposed framework.

\vskip 2mm

\noindent {\bf Keywords}.
principal component analysis, meta-analysis, omics data


%\section{ First-level heading}
%The C98 head 1 style leaves a half-line spacing below a
%first-level heading. There should be one blank line above
%a first-level heading.
%        
%\subsection { Second-level heading}
%There should also be one blank line above a second- or
%third-level heading (but no extra space below them).
%
%Do not intent the first paragraph following a heading.
%Second and subsequent paragraphs are indented by one Tab
%character (= 3 mm). If footnotes are used, they should be
%placed at the foot of the page\footnote{ Footnotes are separated
%from the text by a blank line and a printed line of length 3.5 cm.
%They should be printed in 9-point Times Roman in single line spacing.}.
%        
%\subsubsection { Third-level heading}
%Please specify references using the conventions
%illustrated below. Each should begin on a new line, and
%second and subsequent lines should be on the same page
%indented by 3 mm.

\subsection*{References}

\begin{description}

\item
Flury (1984)
\textit{Common principal components in k groups.}
Journal of the American Statistical Association, 79, 892--898.
\item
Krzanowski (1979)
\textit{Between-groups comparison of principal components.}
Journal of the American Statistical Association, 74, 703--707

\end{description}

\end{document}






