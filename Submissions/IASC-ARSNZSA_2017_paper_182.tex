\documentclass[12pt]{article}
% \documentstyle{iascars2017}

% \usepackage{iascars2017}

\pagestyle{myheadings}
\pagenumbering{arabic}
\topmargin 0pt \headheight 23pt \headsep 24.66pt
%\topmargin 11pt \headheight 12pt \headsep 13.66pt
\parindent = 3mm


\begin{document}


\begin{flushleft}


{\LARGE\bf Three-Dimensional Data Visualization Education with Virtual Reality}


\vspace{1.0cm}

J. E. Lee$^1$, S. Ahn$^1$ and D. H. Jang$^1$

\begin{description}

\item $^1 \;$ Department of Statistics, Pukyong National University, 45, Yongso-ro, Nam-gu,
Busan 48513, Korea. E-mail: dhjang@pknu.ac.kr

\end{description}

\end{flushleft}

%  ***** ADD ENOUGH VERTICAL SPACE HERE TO ENSURE THAT THE *****
%  ***** ABSTRACT (OR MAIN TEXT) STARTS 5 CM BELOW THE TOP *****

\vspace{0.75cm}

\noindent {\bf Abstract}. A variety of data visualization methods are utilizing to analyze huge amount of data. Among various methods, a three-dimensional image requires the rotation of the image to show stereo image on the two-dimensional screen. This study discusses data visualization education of two methods (static method and dynamic method) which make it possible to analyze the construct of stereo image to improve the restriction of the three-dimensional image display with virtual reality. This investigation can be useful to explore three-dimensional data structure more clearly.

\vskip 2mm

\noindent {\bf Keywords}.
Data visualization education, Virtual reality, Stereo image, R package


%\section{ First-level heading}
%The C98 head 1 style leaves a half-line spacing below a
%first-level heading. There should be one blank line above
%a first-level heading.
%
%\subsection { Second-level heading}
%There should also be one blank line above a second- or
%third-level heading (but no extra space below them).
%
%Do not intent the first paragraph following a heading.
%Second and subsequent paragraphs are indented by one Tab
%character (= 3 mm). If footnotes are used, they should be
%placed at the foot of the page\footnote{ Footnotes are separated
%from the text by a blank line and a printed line of length 3.5 cm.
%They should be printed in 9-point Times Roman in single line spacing.}.
%
%\subsubsection { Third-level heading}
%Please specify references using the conventions
%illustrated below. Each should begin on a new line, and
%second and subsequent lines should be on the same page
%indented by 3 mm.

\subsection*{References}

\begin{description}

\item
Bowman, A. (2015). \textit{rpanel}: Simple interactive controls for R using the tcltk library. R package version 1.1-3.

\item
Campos, M. M. (2007). Way Cooler: PCA and Visualization Linear Algebra in the Oracle Database 2, http://oracledmt.blogspot.kr/2007/06/way-cooler-pca-and-visualization-linear.html.

\item
Ligges, U. (2017). \textit{scatterplot3d}: 3D Scatter Plot. R package version 0.3-38.

\item
Murdoch, D. (2017). \textit{rgl}: 3D Visualization Using OpenGL. R package version 0.97.0.

\item
Myers, R. H., Montgomery, D. C. and Anderson-Cook, C. M. (2016). \textit{Response Surface Methodology: Process and Product Optimization Using Designed Experiments, 4th ed}, Wiley, New York.

\item
Ripley, B. (2016). \textit{MASS}: Support Functions and Datasets for Venables and Ripley's MASS. R package version 7.3-47.

\item
Sarkar, D. (2016). \textit{lattice}: Trellis Graphics for R. R package version 0.20-35.

\item
Soetaert, K. (2016). \textit{plot3D}: Plotting Multi-Dimensional Data. R package version 1.1.

\item
Wolf, H. P. (2015). \textit{aplpack}: Another Plot PACKage: stem.leaf, bagplot, faces, spin3R, plotsummary, plothulls, and some slider functions. R package version 1.3.0.

\item
http://astrostatistics.psu.edu/datasets/SDSS quasar.html.

\item
http://forbes.com/mlb valuations/list.

\item
http://gartner.com/newsroom/id/3412017.

\end{description}

\end{document}





