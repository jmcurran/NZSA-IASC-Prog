\documentclass[12pt]{article}
% \documentstyle{iascars2017}

% \usepackage{iascars2017}

\pagestyle{myheadings} 
\pagenumbering{arabic}
\topmargin 0pt \headheight 23pt \headsep 24.66pt
%\topmargin 11pt \headheight 12pt \headsep 13.66pt
\parindent = 3mm 


\begin{document}


\begin{flushleft}


{\LARGE\bf Bayesian analysis for fitting zero-inflated count data with data augmentation}


\vspace{1.0cm}

Beom Seuk Hwang$^1$ and Zhen Chen$^2$

\begin{description}

\item $^1 \;$ Department of Applied Statistics, Chung-Ang University,
Seoul, Korea

\item $^2 \;$ Eunice Kennedy Shriver National Institute of Child Health and Human Development, MD, 20892 USA

\end{description}

\end{flushleft}

%  ***** ADD ENOUGH VERTICAL SPACE HERE TO ENSURE THAT THE *****
%  ***** ABSTRACT (OR MAIN TEXT) STARTS 5 CM BELOW THE TOP *****

\vspace{0.75cm}

\noindent {\bf Abstract}. Count data with excess zeros are common in epidemiological studies. Zero-inflated Poisson (ZIP) model or zero-inflated negative binomial (ZINB) model can be usually used in these cases. From Bayesian perspective, however, the ZIP and ZINB models are not straightforward to fit, usually requiring manual tunings in the Markov chain Monte Carlo algorithm. We consider the auxiliary mixture sampling through several data augmentations that effectively transform the non-linear and non-Gaussian problem in zero-inflated regression model into a set of linear and Gaussian one. The auxiliary mixture sampling results in tuning-free algorithms in MCMC. We demonstrate how the auxiliary mixture sampling can be applied to an epidemiological case study. 

\vskip 2mm

\noindent {\bf Keywords}.
Auxiliary mixture sampling, ZIP model, ZINB model, Markov chain Monte Carlo


%\section{ First-level heading}
%The C98 head 1 style leaves a half-line spacing below a
%first-level heading. There should be one blank line above
%a first-level heading.
%        
%\subsection { Second-level heading}
%There should also be one blank line above a second- or
%third-level heading (but no extra space below them).
%
%Do not intent the first paragraph following a heading.
%Second and subsequent paragraphs are indented by one Tab
%character (= 3 mm). If footnotes are used, they should be
%placed at the foot of the page\footnote{ Footnotes are separated
%from the text by a blank line and a printed line of length 3.5 cm.
%They should be printed in 9-point Times Roman in single line spacing.}.
%        
%\subsubsection { Third-level heading}
%Please specify references using the conventions
%illustrated below. Each should begin on a new line, and
%second and subsequent lines should be on the same page
%indented by 3 mm.


\end{document}





