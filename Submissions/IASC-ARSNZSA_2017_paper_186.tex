\documentclass[12pt]{article}
% \documentstyle{iascars2017}

% \usepackage{iascars2017}

\pagestyle{myheadings} 
\pagenumbering{arabic}
\topmargin 0pt \headheight 23pt \headsep 24.66pt
%\topmargin 11pt \headheight 12pt \headsep 13.66pt
\parindent = 3mm 


\begin{document}


\begin{flushleft}


{\LARGE\bf Phylogenetic Tree-based Microbiome Association Test}


\vspace{1.0cm}

Sungho Won$^{1,2}$

\begin{description}

\item $^1 \;$ Department of Public Health Science, Seoul National University,
Seoul, Korea

\item $^2 \;$ Interdisciplinary Program for Bioinformatics, College of Natural Science, Seoul National University, Seoul



\end{description}

\end{flushleft}

%  ***** ADD ENOUGH VERTICAL SPACE HERE TO ENSURE THAT THE *****
%  ***** ABSTRACT (OR MAIN TEXT) STARTS 5 CM BELOW THE TOP *****

\vspace{0.75cm}

\noindent {\bf Abstract}. Microbial metagenomics data has large inter-subject variation and operational taxonomic units (OTU) for each species are usually very sparse. Because of these problems, non-parametric approaches such as Mann-Whitney U test and Wilcoxon rank-sum test have been utilized. However these approaches suffer from low statistical powers for association analyses and thus investigation on efficient statistical analyses is necessary. Main goal in my thesis is to propose phylogenetic Tree-based Microbiome Association Test (TMAT) for association analyses between microbiome abundances of each OTU and disease phenotype. Phylogenetic tree reveals similarity between different OTUs, and thus was used to provide TMAT. TMAT calculates score test statistics for each node and test statistics for all nodes are combined into a single statistics by minimum p-value or Fisher's combing p-value method. TMAT was compared with existing methods with extensive simulations. Simulation studies show that TMAT preserves the nominal type-1 error and its statistical powers were usually much better than existing methods for considered scenarios. Furthermore it was applied to atopic diseases and found that community profiles of Enterococcus is associated.
\vskip 2mm

\noindent {\bf Keywords}.
NGS; phylogenetic treel Microbiome Association Test

%\section{ First-level heading}
%The C98 head 1 style leaves a half-line spacing below a
%first-level heading. There should be one blank line above
%a first-level heading.
%        
%\subsection { Second-level heading}
%There should also be one blank line above a second- or
%third-level heading (but no extra space below them).
%
%Do not intent the first paragraph following a heading.
%Second and subsequent paragraphs are indented by one Tab
%character (= 3 mm). If footnotes are used, they should be
%placed at the foot of the page\footnote{ Footnotes are separated
%from the text by a blank line and a printed line of length 3.5 cm.
%They should be printed in 9-point Times Roman in single line spacing.}.
%        
%\subsubsection { Third-level heading}
%Please specify references using the conventions
%illustrated below. Each should begin on a new line, and
%second and subsequent lines should be on the same page
%indented by 3 mm.


\end{document}






