\documentclass[12pt]{article}
% \documentstyle{iascars2017}

% \usepackage{iascars2017}

\pagestyle{myheadings} 
\pagenumbering{arabic}
\topmargin 0pt \headheight 23pt \headsep 24.66pt
%\topmargin 11pt \headheight 12pt \headsep 13.66pt
\parindent = 3mm 


\begin{document}


\begin{flushleft}


{\LARGE\bf Latent variable models and multivariate binomial data}


\vspace{1.0cm}

John Holmes$^1$

\begin{description}

\item $^1 \;$ Department of Mathematics and Statistics, University of Otago,
Dunedin, P.O. Box 56, New Zealand


\end{description}

\end{flushleft}

%  ***** ADD ENOUGH VERTICAL SPACE HERE TO ENSURE THAT THE *****
%  ***** ABSTRACT (OR MAIN TEXT) STARTS 5 CM BELOW THE TOP *****

\vspace{0.75cm}

\noindent {\bf Abstract}. A large body of work has been devoted to latent variable models applicable to multivariate binary data. However little work has been put into extending these models to cases where the observed data is multivariate binomial. In this paper, we will first show that models that use either a logit or probit link function, offer the same level of modelling flexibility in the binary case, but only the logit link fits into a data augmentation approach that compactly extends from binary to binomial. Secondly, we will demonstrate that multivariate binomial data provides greater flexibility in how the link function can be represented. Lastly, we will consider properties of the implied distribution of latent probabilities under a logit link.

\vskip 2mm

\noindent {\bf Keywords}.
Multivariate binomial data, principal components/factor analysis, item response theory, link functions, logit-normal distributions


%\section{ First-level heading}
%The C98 head 1 style leaves a half-line spacing below a
%first-level heading. There should be one blank line above
%a first-level heading.
%        
%\subsection { Second-level heading}
%There should also be one blank line above a second- or
%third-level heading (but no extra space below them).
%
%Do not intent the first paragraph following a heading.
%Second and subsequent paragraphs are indented by one Tab
%character (= 3 mm). If footnotes are used, they should be
%placed at the foot of the page\footnote{ Footnotes are separated
%from the text by a blank line and a printed line of length 3.5 cm.
%They should be printed in 9-point Times Roman in single line spacing.}.
%        
%\subsubsection { Third-level heading}
%Please specify references using the conventions
%illustrated below. Each should begin on a new line, and
%second and subsequent lines should be on the same page
%indented by 3 mm.

\subsection*{References}

\begin{description}

\item
(ed.) Bartholomew, D. J. and Knott, M. and Moustaki, I. (2011).
\textit{Latent Variable Models and Factor Analysis: A Unified Approach}. Chichester: John Wiley \& Sons.

\item
Johnson, N.L. (1949).
Systems of Frequency Curves Generated by Methods of Translation.
\textit{Biometrika}, \textbf{36}, 149--276.

\item
Polson, N. G. and Scott, J. G. and Windle, J. (2013).
Bayesian inference for logistic models using {P{\'o}lya}-gamma latent
	variables.
\textit{Journal of the American Statistical Association}, \textbf{108}, 1339--1349.



\end{description}





\end{document}





