\documentclass[12pt]{article}
% \documentstyle{iascars2017}

% \usepackage{iascars2017}

\pagestyle{myheadings}
\pagenumbering{arabic}
\topmargin 0pt \headheight 23pt \headsep 24.66pt
%\topmargin 11pt \headheight 12pt \headsep 13.66pt
\parindent = 3mm


\begin{document}


\begin{flushleft}


{\LARGE\bf Robust Principal Expectile Component Analysis}


\vspace{1.0cm}

Liang-Ching Lin$^{*1}$, Ray Bing Chen$^1$, Mong-Na Lo Huang$^2$ and Meihui Guo$^2$

\begin{description}

\item $^1  \;$ Department of Statistics, National Cheng Kung University,
Tainan, Taiwan

\item $^2 \;$ Department of Applied Mathematics, National Sun Yat-sen University, Kaohsiung, Taiwan

\end{description}

\end{flushleft}


%  ***** ADD ENOUGH VERTICAL SPACE HERE TO ENSURE THAT THE *****
%  ***** ABSTRACT (OR MAIN TEXT) STARTS 5 CM BELOW THE TOP *****


\vspace{0.75cm}

\noindent {\bf Abstract}. Principal component analysis (PCA) is widely used in dimensionality
reduction for high-dimensional data. It identifies principal components by sequentially
maximizing the component score variance around the mean. However,
in many applications, one is interested in capturing the tail variations of the
data rather than variation around the center. To capture the tail characteristics,
Tran et al. (2016), based on an asymmetric $L_2$ norm, proposed principle expectile
components (PECs). In this study, we introduce a new method called Huber-type
principal expectile component (HPEC) using an asymmetric Huber norm
to produce robust PECs. The statistical properties of HPECs are derived, and
a derivative free optimization approach, particle swarm optimization (PSO), is
used to find HPECs. As a demonstration, HPEC is applied to real and simulated
data with encouraging results.

\vskip 2mm

\noindent {\bf Keywords}.
asymmetric norm, expectile, Huber's criterion, particle
swarm optimization, principle component


%\section{ First-level heading}
%The C98 head 1 style leaves a half-line spacing below a
%first-level heading. There should be one blank line above
%a first-level heading.
%
%\subsection { Second-level heading}
%There should also be one blank line above a second- or
%third-level heading (but no extra space below them).
%
%Do not intent the first paragraph following a heading.
%Second and subsequent paragraphs are indented by one Tab
%character (= 3 mm). If footnotes are used, they should be
%placed at the foot of the page\footnote{ Footnotes are separated
%from the text by a blank line and a printed line of length 3.5 cm.
%They should be printed in 9-point Times Roman in single line spacing.}.
%
%\subsubsection { Third-level heading}
%Please specify references using the conventions
%illustrated below. Each should begin on a new line, and
%second and subsequent lines should be on the same page
%indented by 3 mm.

\subsection*{References}

\begin{description}

\item
Tran, N. M., Burdejov\'{a}, P., Osipenko, M. and H\'{a}rdle, W. K. (2016). \textit{Principal Component
Analysis in an Asymmetric Norm.} SFB 649 Discussion Paper 2016-040, Sonderforschungsbereich
649, Humboldt Universit\'{a}t zu Berlin, Germany.

\end{description}

\end{document}





