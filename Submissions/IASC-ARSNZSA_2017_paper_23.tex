\documentclass[12pt]{article}
% \documentstyle{iascars2017}

% \usepackage{iascars2017}

\pagestyle{myheadings} 
\pagenumbering{arabic}
\topmargin 0pt \headheight 23pt \headsep 24.66pt
%\topmargin 11pt \headheight 12pt \headsep 13.66pt
\parindent = 3mm 


\begin{document}


\begin{flushleft}


{\LARGE\bf Estimation of a High-Dimensional Covariance Matrix}


\vspace{1.0cm}

Xiangjie Xue$^*$ and Yong Wang$^*$

\begin{description}

\item $^* \;$ Department of Statistics, The University of Auckland,
New Zealand.

\end{description}

\end{flushleft}

%  ***** ADD ENOUGH VERTICAL SPACE HERE TO ENSURE THAT THE *****
%  ***** ABSTRACT (OR MAIN TEXT) STARTS 5 CM BELOW THE TOP *****

\vspace{0.75cm}

\noindent {\bf Abstract}. The estimation of covariance or precision (inverse covariance) matrices plays a prominent role in multivariate analysis. The usual estimator, the sample covariance matrix, is known to be unstable and ill-conditioned in high-dimensional setting. In the past two decades, various methods have been developed to give a stable and well-conditioned estimator and they have their own advantages and disadvantages. We will review some of the most popular methods \nocite{RN36} and describe a new method to estimate the correlation matrix and hence the covariance matrix using the empirical Bayes method. Similar to many element-wise methods in the literature, we also assume that the elements in a correlation matrix are independent of each other. We use the fact that the elements in a sample correlation matrix can be approximated by the same one-parameter normal distribution with unknown means \nocite{RN31}, along with the non-parametric maximum likelihood estimation \nocite{RN87} to give a new estimator of the correlation matrix. Preliminary simulation results show that the new estimator has some advantages over various thresholding methods in estimating sparse covariance matrices.



\vskip 2mm

\noindent {\bf Keywords}.
Big Data, Multivariate Analysis, Statistical Inference


%\section{ First-level heading}
%The C98 head 1 style leaves a half-line spacing below a
%first-level heading. There should be one blank line above
%a first-level heading.
%        
%\subsection { Second-level heading}
%There should also be one blank line above a second- or
%third-level heading (but no extra space below them).
%
%Do not intent the first paragraph following a heading.
%Second and subsequent paragraphs are indented by one Tab
%character (= 3 mm). If footnotes are used, they should be
%placed at the foot of the page\footnote{ Footnotes are separated
%from the text by a blank line and a printed line of length 3.5 cm.
%They should be printed in 9-point Times Roman in single line spacing.}.
%        
%\subsubsection { Third-level heading}
%Please specify references using the conventions
%illustrated below. Each should begin on a new line, and
%second and subsequent lines should be on the same page
%indented by 3 mm.

\subsection*{References}

\begin{description}
	
	\item
	Efron, B., 2010. 
	\textit{Correlated $z$-values and the accuracy of large-scale statistical estimates}.
	J Am Stat Assoc \textbf{105}, 1042 - 1055.
	
	\item
	Fan, J., Liao, Y., Liu, H., 2016. 
	\textit{An overview of the estimation of large covariance and
	precision matrices}. Econometrics Journal \textbf{19}, C1 - C32.
	
	\item
	Wang, Y., 2007. 
	\textit{On fast computation of the non-parametric maximum likelihood estimate
	of a mixing distribution}. 
	Journal of the Royal Statistical Society: Series B \textbf{69}, 185 - 198.

	\end{description}
\end{document}





