\documentclass[12pt]{article}
% \documentstyle{iascars2017}

% \usepackage{iascars2017}

\pagestyle{myheadings} 
\pagenumbering{arabic}
\topmargin 0pt \headheight 23pt \headsep 24.66pt
%\topmargin 11pt \headheight 12pt \headsep 13.66pt
\parindent = 3mm 


\begin{document}


\begin{flushleft}


{\LARGE\bf Adaptive Model Averaging in High-Dimensional Linear Regression}


\vspace{1.0cm}

Wei-Cheng Hsiao$^1$, Ching-Kang Ing$^2$, and Tzu-Chang Forrest Cheng$^3$

\begin{description}

\item $^1 \;$ Institute of Statistics, National Tsing-Hua University, Hsinchu 30013, Taiwan 

\item $^2 \;$ Institute of Statistics, National Tsing-Hua University, Hsinchu 30013, Taiwan 


\item $^3 \;$ Department of Economics, National Central University, Taoyuan 32001, Taiwan
\end{description}

\end{flushleft}

%  ***** ADD ENOUGH VERTICAL SPACE HERE TO ENSURE THAT THE *****
%  ***** ABSTRACT (OR MAIN TEXT) STARTS 5 CM BELOW THE TOP *****

\vspace{0.75cm}

\noindent {\bf Abstract}. This paper aims to propose a data-adaptive model averaging estimation in the high-dimensional framework. To this end, We first consider the orthogonal greedy algorithm (OGA) proposed by Ing and Lai (2011) to construct a set of nested models. The high-dimensional model averaging criteria (HDMMA) suggested by Ing (2016) is considered upon the OGA nested models, while the penalty term is unknown and needed to be estimated. We then use the cross-validation to select the optimal penalty. This method of penalty selection is shown to be optimally adaptive to a wide class of data generating processes. Furthermore, the resultant HDMMA estimator based on the selected penalty is shown to be asymptotic rate efficient. Finally, numerical studies in this paper are expected to shed some light on the choice of data splitting ratio for the cross-validation. 

\vskip 2mm

\noindent {\bf Keywords}.
Adaptive penalty, Cross-validation, High dimension, Model averaging, Greedy algorithm


%\section{ First-level heading}
%The C98 head 1 style leaves a half-line spacing below a
%first-level heading. There should be one blank line above
%a first-level heading.
%        
%\subsection { Second-level heading}
%There should also be one blank line above a second- or
%third-level heading (but no extra space below them).
%
%Do not intent the first paragraph following a heading.
%Second and subsequent paragraphs are indented by one Tab
%character (= 3 mm). If footnotes are used, they should be
%placed at the foot of the page\footnote{ Footnotes are separated
%from the text by a blank line and a printed line of length 3.5 cm.
%They should be printed in 9-point Times Roman in single line spacing.}.
%        
%\subsubsection { Third-level heading}
%Please specify references using the conventions
%illustrated below. Each should begin on a new line, and
%second and subsequent lines should be on the same page
%indented by 3 mm.
\bibliography{C:/Users/superuser/Dropbox/Documents/tex/reference}
\subsection*{References}

\begin{description}

\item
Ing, C.-K. and Lai, T. L. (2011). A stepwise regression method and consistent model selection
for high-dimensional sparse linear models. \textit{Statist. Sinica}, \textbf{21}, 1473--1513.
\item
Ing, C.-K. (2016). Model averaging in high-dimensional regressions. \textit{Unpublished Technical Report}.

\end{description}

\end{document}





