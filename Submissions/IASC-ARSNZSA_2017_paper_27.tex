\documentclass[12pt]{article}
% \documentstyle{iascars2017}

% \usepackage{iascars2017}

\pagestyle{myheadings}
\pagenumbering{arabic}
\topmargin 0pt \headheight 23pt \headsep 24.66pt
%\topmargin 11pt \headheight 12pt \headsep 13.66pt
\parindent = 3mm


\begin{document}


\begin{flushleft}


{\LARGE\bf Fluctuation Reduction of Value-at-Risk Estimation and its Applications}


\vspace{1.0cm}

Shih-Feng Huang

\begin{description}
\item
Department of Applied Mathematics, National University of Kaohsiung,
Kaohsiung, 81148, Taiwan

\end{description}

\end{flushleft}

%  ***** ADD ENOUGH VERTICAL SPACE HERE TO ENSURE THAT THE *****
%  ***** ABSTRACT (OR MAIN TEXT) STARTS 5 CM BELOW THE TOP *****

\vspace{0.75cm}

\noindent {\bf Abstract}. 
 Value-at-Risk (VaR) is a fundamental tool for risk management and is also
 associated with the capital requirements of banks. Banks need to adjust their
 capital levels for satisfying the Basel Capital Accord. On the other hand,
 managements do not like to change the capital levels too often. To achieve
 a balance, this study proposes an approach to reduce the fluctuation of VaR
 estimation. The first step is to fit a time series model to the underlying asset returns
 and obtain the conventional VaR process. A new VaR (NVaR)
 estimation of the conventional VaR process is then determined by applying
 change-point detection algorithms and a proposed combination scheme. The
 capital levels computed from the NVaR process are capable of satisfying the
 Basel Accord and reducing the fluctuation of capital levels simultaneously. To
 apply the proposed method to the calculation of future capital requirements,
 an innovative approach for NVaR prediction is also proposed by incorporating
 the concept of CUSUM control charts. The return processes of 30 companies
 on the list of S$\&$P 500 from 2005 to 2016 are employed for our empirical investigation.
 Numerical results indicate that the proposed NVaR prediction is
 capable of satisfying the Basel Accord and reducing the fluctuation of capital
 requirements simultaneously by using a comparable average amount of capital
 requirements to the conventional VaR estimator.


\vskip 2mm

\noindent {\bf Keywords}.
Capital requirement, change point detection, CUSUM control chart, fluctuation reduction, Value-at-Risk


%\section{ First-level heading}
%The C98 head 1 style leaves a half-line spacing below a
%first-level heading. There should be one blank line above
%a first-level heading.
%
%\subsection { Second-level heading}
%There should also be one blank line above a second- or
%third-level heading (but no extra space below them).
%
%Do not intent the first paragraph following a heading.
%Second and subsequent paragraphs are indented by one Tab
%character (= 3 mm). If footnotes are used, they should be
%placed at the foot of the page\footnote{ Footnotes are separated
%from the text by a blank line and a printed line of length 3.5 cm.
%They should be printed in 9-point Times Roman in single line spacing.}.
%
%\subsubsection { Third-level heading}
%Please specify references using the conventions
%illustrated below. Each should begin on a new line, and
%second and subsequent lines should be on the same page
%indented by 3 mm.
%
%\subsection*{References}
%
%\begin{description}
%\item
%Hawkins, D.M., Qiu, P. and Kang, C.W. (2003).
%\textit{J. Quality Technology}, \textbf{35}, 355--366.
%
%\item
%Korkas, K. and Fryzlewicz, P. (2017).
%\textit{Statistica Sinica}, \textbf{27}, 287--311.
%
%\item
%Matteson, D.S. and James, N.A. (2014).
%\textit{J. American Statistical Association}, \textbf{109}, 334--345.
%
%\item
%Montgomery, D.C. (2013).
%\textit{Introduction to Statistical Quality Control: A Modern Introduction: 7th Edition.}
%New Jersey: John Wiley \& Sons.
%
%\item
%Shi, X., Wang, X.S., Wei, D. and Wu, Y. (2015).
%\textit{Computational Statistics}, \textbf{31}, 671--691.
%\end{description}

\end{document}





