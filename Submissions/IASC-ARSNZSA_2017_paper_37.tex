\documentclass[12pt]{article}
% \documentstyle{iascars2017}

% \usepackage{iascars2017}

\pagestyle{myheadings} 
\pagenumbering{arabic}
\topmargin 0pt \headheight 23pt \headsep 24.66pt
%\topmargin 11pt \headheight 12pt \headsep 13.66pt
\parindent = 3mm 


\begin{document}


\begin{flushleft}


{\LARGE\bf Bayesian curve fitting for discontinuous function using overcomplete representation with multiple kernels}


\vspace{1.0cm}

Youngseon Lee$^1$, Shuhei Mano$^2$ and Jaeyong Lee$^1$

\begin{description}

\item $^1 \;$ Department of Statistics, College of Natural Science, Seoul National University, 56-1 Mountain, Sillim-dong, Gwanak-gu, Seoul, Korea

\item $^2 \;$ The Institute of Statistical Mathematics, 
10-3 Midori-cho, Tachikawa, Tokyo 190-8562, Japan

\end{description}

\end{flushleft}

%  ***** ADD ENOUGH VERTICAL SPACE HERE TO ENSURE THAT THE *****
%  ***** ABSTRACT (OR MAIN TEXT) STARTS 5 CM BELOW THE TOP *****

\vspace{0.75cm}

\noindent {\bf Abstract}. We propose a new Bayesian methodology for estimating discontinuous functions. In this model, the estimated function is expressed by the overcomplete representation with multiple kernels. Therefore, the complex shape of functions can be expressed by the much smaller number of parameters due to the nature of the sparseness. It does not need any assumptions about the location of discontinuities, the smoothness of the function, the number of features. The form of the function taking all of these into account is determined naturally by the random Levy measure. Simulation data and real data analysis show that this model is suitable for fitting discontinuous functions. We also proved theoretical properties about the support of the function space having jumps in this paper.

\vskip 2mm

\noindent {\bf Keywords}.
Bayesian, nonparametric regression, discontinuous curve fitting, overcomplete, multiple kernel, Levy random field


%\section{ First-level heading}
%The C98 head 1 style leaves a half-line spacing below a
%first-level heading. There should be one blank line above
%a first-level heading.
%        
%\subsection { Second-level heading}
%There should also be one blank line above a second- or
%third-level heading (but no extra space below them).
%
%Do not intent the first paragraph following a heading.
%Second and subsequent paragraphs are indented by one Tab
%character (= 3 mm). If footnotes are used, they should be
%placed at the foot of the page\footnote{ Footnotes are separated
%from the text by a blank line and a printed line of length 3.5 cm.
%They should be printed in 9-point Times Roman in single line spacing.}.
%        
%\subsubsection { Third-level heading}
%Please specify references using the conventions
%illustrated below. Each should begin on a new line, and
%second and subsequent lines should be on the same page
%indented by 3 mm.

\subsection*{References}

\begin{description}
	

\item
Chu, J. H., Clyde, M. A., and Liang, F. (2009).
Bayesian function estimation using continuous wavelet dictionaries,
\textit{Statistica Sinica}, 1419--1438

\item
Clyde, M. A., and Wolpert, R. L. (2007).
Nonparametric function estimation using overcomplete dictionaries,
\textit{Bayesian Statistics}, 
\textbf{8}, 91--114.


\item
Green, Peter J. (1995).
Reversible jump Markov chain Monte Carlo computation and Bayesian model determination,
\textit{Biometrika}, \textbf{82(4)}, 711--732.


\item 
Khinchine, Alexander Ya and L{\'e}vy, Paul (1936).
Sur les lois stables, \textit{CR Acad. Sci. Paris}, 
\textbf{202}, 374--376.


\item 
M{\"u}ller, P., and Quintana, F. A. (2004).
Nonparametric Bayesian data analysis,
\textit{Statistical science}, 95--110


\item 
Pillai, N. S., Wu, Q., Liang, F., Mukherjee, S., and Wolpert, R. L. (2007). 
Characterizing the function space for Bayesian kernel models,
\textit{Journal of Machine Learning Research}, 
\textbf{8}, 1769--1797.

\item 
Qiu, Peihua (2011).
\textit{Jump Regression Analysis}.
Springer.


\item  
Wolpert, R. L., Clyde, M. A., and Tu, C. (2011).
Stochastic expansions using continuous dictionaries: L{\'e}vy adaptive regression kernels, The \textit{Annals of Statistics},
1916--1962.


\end{description}

\end{document}





