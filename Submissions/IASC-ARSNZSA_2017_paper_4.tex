\documentclass[12pt]{article}
% \documentstyle{iascars2017}

% \usepackage{iascars2017}

\pagestyle{myheadings} 
\pagenumbering{arabic}
\topmargin 0pt \headheight 23pt \headsep 24.66pt
%\topmargin 11pt \headheight 12pt \headsep 13.66pt
\parindent = 3mm 


\begin{document}


\begin{flushleft}


{\LARGE\bf  The joint models for nonlinear longitudinal and time-to-event data using penalized splines: A Bayesian approach}


\vspace{1.0cm}

Pham Thi Thu Huong$^1$ and  Darfiana Nur$^1$ and  Alan Branford$^1$

\begin{description}

\item $^1 \;$  The College of Science and Engineering, Flinders University of Australia at Tonsley, 1284 South Road, Tonsley, SA 5042, Australia

\end{description}

\end{flushleft}

%  ***** ADD ENOUGH VERTICAL SPACE HERE TO ENSURE THAT THE *****
%  ***** ABSTRACT (OR MAIN TEXT) STARTS 5 CM BELOW THE TOP *****

\vspace{0.65cm}

\noindent {\bf Abstract}. 
The joint models for longitudinal data and time-to-event data have been introduced to measure the association between longitudinal data and survival time in
clinical, epidemiological and educational studies.. 
The main aim of this talk is to estimate the parameters in the joint models using a Bayesian approach for nonlinear longitudinal data and time-to-event data using penalized splines. To perform this analysis, the joint posterior distribution of hazard rate at baseline, survival
and longitudinal coefficient and random effects parameters is first being introduced followed by derivation of the conditional
posterior distributions for each of parameter.  Based on these target posterior distributions, the
samples of parameters are simulated using Metropolis, Metropolis Hastings and Gibbs sampler algorithms. An R program is written to implement the analysis.
Finally, the prior sensitivity analysis for the baseline hazard rate and association parameters is performed
following by simulations studies and a case study.

\vskip 2mm

\noindent {\bf Keywords}.
Bayesian analysis, Joint models, Longitudinal data, MCMC algorithms, Prior sensitivity analysis, Survival data

%\section{ First-level heading}
%The C98 head 1 style leaves a half-line spacing below a
%first-level heading. There should be one blank line above
%a first-level heading.
%        
%\subsection { Second-level heading}
%There should also be one blank line above a second- or
%third-level heading (but no extra space below them).
%
%Do not intent the first paragraph following a heading.
%Second and subsequent paragraphs are indented by one Tab
%character (= 3 mm). If footnotes are used, they should be
%placed at the foot of the page\footnote{ Footnotes are separated
%from the text by a blank line and a printed line of length 3.5 cm.
%They should be printed in 9-point Times Roman in single line spacing.}.
%        
%\subsubsection { Third-level heading}
%Please specify references using the conventions
%illustrated below. Each should begin on a new line, and
%second and subsequent lines should be on the same page
%indented by 3 mm.

\subsection*{References}

\begin{description}

\item
D. Rizopoulos, D. (2014).  The R package JMbayes for fitting joint models for longitudinal and time-to- 
event data using MCMC.
\textit{Journal of Statistical Software,} 72(7):1 -- 45.

\item
Brown, E. R., J. G. Ibrahim, J. G., DeGruttola, V. (2005). A flexible B-spline model for multiple 
longitudinal biomarkers and survival.\textit{Biometrics,} 61(1):64 -- 73.


\end{description}

\end{document}


