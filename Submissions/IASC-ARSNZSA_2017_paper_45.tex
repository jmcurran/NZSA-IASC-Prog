\documentclass[12pt]{article}
% \documentstyle{iascars2017}

% \usepackage{iascars2017}

\pagestyle{myheadings} 
\pagenumbering{arabic}
\topmargin 0pt \headheight 23pt \headsep 24.66pt
%\topmargin 11pt \headheight 12pt \headsep 13.66pt
\parindent = 3mm 


\begin{document}


\begin{flushleft}


{\LARGE\bf Mixed models for complex survey data}


\vspace{1.0cm}

Xudong Huang $^1$ and Thomas Lumley$^2$

\begin{description}

\item $^1 \;$ Department of Statistics, University of Auckland,
New Zealand

\item $^2 \;$ Department of Statistics, University of Auckland,
New Zealand

\end{description}

\end{flushleft}

%  ***** ADD ENOUGH VERTICAL SPACE HERE TO ENSURE THAT THE *****
%  ***** ABSTRACT (OR MAIN TEXT) STARTS 5 CM BELOW THE TOP *****

\vspace{0.75cm}

\noindent {\bf Abstract}. I want to fit a mixed model to a population distribution, but I have data from a complex (multistage) sample. The sampling is informative, that is, the model holding for the population is different from the model holding for the (biased) sample. Ignoring the sampling design and just fitting the mixed model to the sample distribution will lead to biased inference. Although both the model and sampling involve “clusters”, the model clusters and sample clusters need not be the same.  I will use a pairwise composite likelihood method to estimate the parameters of the population model under this setting. In particular, consistency and asymptotic normality can be established. Variance estimation in this problem is challenging. I will talk about a variance estimator and how to show it is consistent. 


\vskip 2mm

\noindent {\bf Keywords}.
Mixed model, Complex sampling, Pairwise composite likelihood


%\section{ First-level heading}
%The C98 head 1 style leaves a half-line spacing below a
%first-level heading. There should be one blank line above
%a first-level heading.
%        
%\subsection { Second-level heading}
%There should also be one blank line above a second- or
%third-level heading (but no extra space below them).
%
%Do not intent the first paragraph following a heading.
%Second and subsequent paragraphs are indented by one Tab
%character (= 3 mm). If footnotes are used, they should be
%placed at the foot of the page\footnote{ Footnotes are separated
%from the text by a blank line and a printed line of length 3.5 cm.
%They should be printed in 9-point Times Roman in single line spacing.}.
%        
%\subsubsection { Third-level heading}
%Please specify references using the conventions
%illustrated below. Each should begin on a new line, and
%second and subsequent lines should be on the same page
%indented by 3 mm.

\subsection*{References}

\begin{description}

\item
Yi, G. ,  Rao, J.  and  Li, H.(2016).
\textit{A weighted composite likelihood approach for analysis of survey data under two-level models.}
Statistica Sinica, 2016, 26, 569-587


\end{description}

\end{document}





