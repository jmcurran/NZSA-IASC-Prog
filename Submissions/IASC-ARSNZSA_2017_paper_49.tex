\documentclass[12pt]{article}
% \documentstyle{iascars2017}

% \usepackage{iascars2017}

\pagestyle{myheadings}
\pagenumbering{arabic}
\topmargin 0pt \headheight 23pt \headsep 24.66pt
%\topmargin 11pt \headheight 12pt \headsep 13.66pt
\parindent = 3mm


\begin{document}


\begin{flushleft}


{\LARGE\bf Bayesian Structure Selection for Vector Autoregression
Model}


\vspace{1.0cm}

Chi-Hsiang Chu$^1$, Mong-Na Lo Huang$^1$, Shih-Feng Huang$^2$ and
Ray-Bing Chen$^3$

\begin{description}

\item $^1 \;$ Department of Applied Mathematics, National Sun Yat-sen University, Kaohsiung,
Taiwan

\item $^2 \;$ Department of Applied Mathematics, National University of Kaohsiung,
Kaohsiung, Taiwan

\item $^3 \;$ Department of Statistics, National Cheng Kung University, Tainan,
Taiwan

\end{description}

\end{flushleft}

%  ***** ADD ENOUGH VERTICAL SPACE HERE TO ENSURE THAT THE *****
%  ***** ABSTRACT (OR MAIN TEXT) STARTS 5 CM BELOW THE TOP *****

\vspace{0.75cm}

\noindent {\bf Abstract}. Vector autoregression (VAR) model is
powerful in economic data analysis because it can be used to analyze
several different time series data simultaneously. However, in VAR
model, we need to deal with the huge coefficient dimensionality and
it would be caused some computational problems for coefficient
inference. To reduce the dimensionality, we could take some model
structures into account based on the prior knowledge. In this paper,
several group structures of the coefficient matrices are considered.
Due to different types of VAR structures, corresponding MCMC
algorithms are proposed to generate posterior samples for making
inference of the structure selection. Simulation studies and a real
example are used to show the performances of the proposed Bayesian
approaches.

\vskip 2mm

\noindent {\bf Keywords}. Bayesian variable selection, time series,
universal grouping, segmentized grouping


%\section{ First-level heading}
%The C98 head 1 style leaves a half-line spacing below a
%first-level heading. There should be one blank line above
%a first-level heading.
%
%\subsection { Second-level heading}
%There should also be one blank line above a second- or
%third-level heading (but no extra space below them).
%
%Do not intent the first paragraph following a heading.
%Second and subsequent paragraphs are indented by one Tab
%character (= 3 mm). If footnotes are used, they should be
%placed at the foot of the page\footnote{ Footnotes are separated
%from the text by a blank line and a printed line of length 3.5 cm.
%They should be printed in 9-point Times Roman in single line spacing.}.
%
%\subsubsection { Third-level heading}
%Please specify references using the conventions
%illustrated below. Each should begin on a new line, and
%second and subsequent lines should be on the same page
%indented by 3 mm.

%\subsection*{References}
%
%\begin{description}
%
%\item
%Chen, R.-B., Chu, C. H., Lai, T. Y. and Wu, Y. N. (2011). Stochastic
%matching pursuit for Bayesian variable selection. \emph{Statistics
%and Computing}, {\bf 21}, 247-289.
%
%
%\item
%Chen, R.-B., Chu, C. H., Lai, T. Y. and Wu, Y. N. (2016).
% Bayesian Sparse Group Selection.
% \emph{Journal of Computational and Graphical Statistics}, {\bf 25},
% 665-683.
%
%\item
%Song, S. and Bickel, P. J. (2011). Large vector auto regressions.
%Preprint.
%
%\item
%Nicholson, W., Matteson, D. and Bien, J. (2014). Structured
%regularization for large vector autoregressions. Preprint.
%

%\end{description}

\end{document}
