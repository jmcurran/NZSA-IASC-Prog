\documentclass[12pt]{article}
% \documentstyle{iascars2017}

% \usepackage{iascars2017}

\pagestyle{myheadings} 
\pagenumbering{arabic}
\topmargin 0pt \headheight 23pt \headsep 24.66pt
%\topmargin 11pt \headheight 12pt \headsep 13.66pt
\parindent = 3mm 


\begin{document}


\begin{flushleft}


{\LARGE\bf Improving the production cycle at Stats NZ with RStudio}


\vspace{1.0cm}

Gareth Minshall$^1$ and Chris Hansen$^1$ 

\begin{description}

\item $^1 \;$ Stats NZ, New Zealand

\end{description}

\end{flushleft}

%  ***** ADD ENOUGH VERTICAL SPACE HERE TO ENSURE THAT THE *****
%  ***** ABSTRACT (OR MAIN TEXT) STARTS 5 CM BELOW THE TOP *****

\vspace{0.75cm}

\noindent {\bf Abstract}. Stats  NZ  are  looking  to  move  away  from  the  collection  and  publication  of  stand-alone surveys  to  making  use  of a  wide  range  of  data  sources  and  estimation  strategies.  A key component to enabling this change is to develop the infrastructure which allows analysts to explore, test and use a range of tools which are not traditionally heavily used within National Statistics Offices.  One of the tools Stats NZ is looking to make heavier use of is R.  This talk will outline the development of internal RStudio and Shiny servers at Stats NZ, and give examples demonstrating the types of innovation RStudio has enabled at Stats NZ to improve the way we produce and disseminate statistics.

\vskip 2mm

\noindent {\bf Keywords}.
Shiny, R Markdown, Official Statistics

\vskip 2mm

\noindent {\bf Acknowledgement}. This work was supported by JSPS KAKENHI Grant Number JP16H02013.


%\section{ First-level heading}
%The C98 head 1 style leaves a half-line spacing below a
%first-level heading. There should be one blank line above
%a first-level heading.
%        
%\subsection { Second-level heading}
%There should also be one blank line above a second- or
%third-level heading (but no extra space below them).
%
%Do not intent the first paragraph following a heading.
%Second and subsequent paragraphs are indented by one Tab
%character (= 3 mm). If footnotes are used, they should be
%placed at the foot of the page\footnote{ Footnotes are separated
%from the text by a blank line and a printed line of length 3.5 cm.
%They should be printed in 9-point Times Roman in single line spacing.}.
%        
%\subsubsection { Third-level heading}
%Please specify references using the conventions
%illustrated below. Each should begin on a new line, and
%second and subsequent lines should be on the same page
%indented by 3 mm.

\end{document}





