\documentclass[12pt]{article}
% \documentstyle{iascars2017}

% \usepackage{iascars2017}

\pagestyle{myheadings} 
\pagenumbering{arabic}
\topmargin 0pt \headheight 23pt \headsep 24.66pt
%\topmargin 11pt \headheight 12pt \headsep 13.66pt
\parindent = 3mm 


\begin{document}


\begin{flushleft}


{\LARGE\bf R - A powerful analysis tool to improve Official Statistics in Romania}


\vspace{1.0cm}

Nicoleta Caragea$^1$ and Antoniade Ciprian Alexandru$^2$

\begin{description}

\item $^1 \;$ National Institute of Statistics / Ecological University of Bucharest, Bucharest, ROMANIA

\item $^2 \;$ Ecological University of Bucharest / National Institute of Statistics, Bucharest, ROMANIA

\end{description}

\end{flushleft}

%  ***** ADD ENOUGH VERTICAL SPACE HERE TO ENSURE THAT THE *****
%  ***** ABSTRACT (OR MAIN TEXT) STARTS 5 CM BELOW THE TOP *****

\vspace{0.75cm}

\noindent {\bf Abstract}. This presentation is focused on how R is used in Romanian official statistics to improve the quality of results provided by different statistical data sources on the base of administrative data.  Some benefits for statistical analysis come when it is possible to link administrative records from different registers together, or when they can be linked with censuses or sample surveys. Many of these record linkage or matching methods must be done under statistically conditions, R program being one of the most powerful analysis tool.
In Romania, there has been increasing attention in recent years to use R in official statistics, through specialized R courses for statisticians and training on the job sessions. A international conference on R (uRos) is yearly organized to provide a public forum for researchers from academia and institutes of statistics. It is also a continuous work to develop statistics based on Big Data, Romania being part of the ESSnet Big Data Project.

\vskip 2mm

\noindent {\bf Keywords}.
R package, data sources, statistics, matching method, linkage method


%\section{ First-level heading}
%The C98 head 1 style leaves a half-line spacing below a
%first-level heading. There should be one blank line above
%a first-level heading.
%        
%\subsection { Second-level heading}
%There should also be one blank line above a second- or
%third-level heading (but no extra space below them).
%
%Do not intent the first paragraph following a heading.
%Second and subsequent paragraphs are indented by one Tab
%character (= 3 mm). If footnotes are used, they should be
%placed at the foot of the page\footnote{ Footnotes are separated
%from the text by a blank line and a printed line of length 3.5 cm.
%They should be printed in 9-point Times Roman in single line spacing.}.
%        
%\subsubsection { Third-level heading}
%Please specify references using the conventions
%illustrated below. Each should begin on a new line, and
%second and subsequent lines should be on the same page
%indented by 3 mm.


\end{document}





