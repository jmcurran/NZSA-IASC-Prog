\documentclass[12pt]{article}
% \documentstyle{iascars2017}

% \usepackage{iascars2017}

\pagestyle{myheadings} 
\pagenumbering{arabic}
\topmargin 0pt \headheight 23pt \headsep 24.66pt
%\topmargin 11pt \headheight 12pt \headsep 13.66pt
\parindent = 3mm 


\begin{document}


\begin{flushleft}


{\LARGE\bf Innovative Bayesian Estimation in the von-Mises Distribution}


\vspace{1.0cm}

Yuta Kamiya$^1$ , Toshinari Kamakura$^2$ and Takemi Yanagimoto$^3$

\begin{description}

\item $^1 \;$ Graduate School of Industrial and  Systems Engineering, Chuo University,
Japan
\vspace{-3mm}
\item $^2 \;$ Department of Industrial  and Systems Engineering, Chuo University, Japan
\vspace{-3mm}
\item $^3 \;$ Institute of Statistical Mathematics, Japan

\end{description}

\end{flushleft}

%  ***** ADD ENOUGH VERTICAL SPACE HERE TO ENSURE THAT THE *****
%  ***** ABSTRACT (OR MAIN TEXT) STARTS 5 CM BELOW THE TOP *****

\vspace{0.75cm}

\noindent {\bf Abstract}. In spite of recent growing interest in applying the von-Mises distribution to circular data in various scientific fields, researches on the parameter estimation are surprisingly sparse.
 The standard estimators are the MLE and the maximum marginal likelihood estimator (Schou 1978).
 Although Bayesian estimators are promising, it looks that they have not been fully developed.
 We propose the posterior mean of the canonical parameter, instead of the mean parameter, under the reference prior.
 This estimator satisfies an optimality property, and performs favorably for wide ranges of true parameters.
 Extensive simulation studies yield that the risks of the proposed estimator are significantly small, compared with the existing estimators.
 An interesting finding is that the estimating function for the dispersion parameter behaves reasonably.
 Notable advantages of the present approach are its straightforward extensions to various procedures, including Bayesian estimator under an informative prior based on the reference prior.
 The proposed estimator is examined by applying to practical datasets.

\vskip 2mm

\noindent {\bf Keywords}.
von-Mises distribution, bayesian estimation, canonical parameter


%\section{ First-level heading}
%The C98 head 1 style leaves a half-line spacing below a
%first-level heading. There should be one blank line above
%a first-level heading.
%        
%\subsection { Second-level heading}
%There should also be one blank line above a second- or
%third-level heading (but no extra space below them).
%
%Do not intent the first paragraph following a heading.
%Second and subsequent paragraphs are indented by one Tab
%character (= 3 mm). If footnotes are used, they should be
%placed at the foot of the page\footnote{ Footnotes are separated
%from the text by a blank line and a printed line of length 3.5 cm.
%They should be printed in 9-point Times Roman in single line spacing.}.
%        
%\subsubsection { Third-level heading}
%Please specify references using the conventions
%illustrated below. Each should begin on a new line, and
%second and subsequent lines should be on the same page
%indented by 3 mm.

\subsection*{References}

\begin{description}

\item
Fisher, Nicholas I. {\it Statistical analysis of circular data.} Cambridge University Press, 1995.

\item
Schou, Geert. "Estimation of the concentration parameter in von Mises–Fisher distributions." Biometrika 65.2 (1978): 369-377.
\end{description}

\end{document}





