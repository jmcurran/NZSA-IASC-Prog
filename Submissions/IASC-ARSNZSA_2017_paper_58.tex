\documentclass[12pt]{article}
% \documentstyle{iascars2017}

% \usepackage{iascars2017}

\pagestyle{myheadings} 
\pagenumbering{arabic}
\topmargin 0pt \headheight 23pt \headsep 24.66pt
%\topmargin 11pt \headheight 12pt \headsep 13.66pt
\parindent = 3mm 


\begin{document}


\begin{flushleft}


{\LARGE\bf Clusterwise low-rank correlation  analysis based on majorization}


\vspace{1.0cm}

Kensuke Tanioka$^{1*}$, Satoru Hiwa$^2$,  \\ Tomoyuki Hiroyasu$^2$  and Hiroshi Yadohisa$^3$

\begin{description}

\item $^1 \;$ Clinical Study Support Center, Wakayama Medical University Hospital,
Wakayama, Japan

\item $^2 \;$ Department of Biomedical Information, Doshisha University, Kyoto, Japan

\item $^3 \;$ Department of Culture and Information Science, Doshisha University,
, Kyoto, Japan

\end{description}

\end{flushleft}

%  ***** ADD ENOUGH VERTICAL SPACE HERE TO ENSURE THAT THE *****
%  ***** ABSTRACT (OR MAIN TEXT) STARTS 5 CM BELOW THE TOP *****

\vspace{0.75cm}

\noindent {\bf Abstract}. 
Given correlation matrices between variables of subjects and these classes of subjects,
it is important to get the distinctive local networks for each class.  
For example, in fMRI data analysis, such the situation is observed. 
In concretely, each correlation matrix between
regions of interests for his/her brain is observed, and 
each information of class is get through the experiment.
In this presentation, to achieve the purpose, we proposed simultaneous analysis
for both clustering of variables and low-rank approximation of correlation matrices corresponding to
each class. 
For the estimation, we adopt the majorization algorithm based on 
Pietersz and Groenen (2004) and Simon and Abell (2010).
Through the proposed method, we can get the distinctive sparse correlation matrices
corresponding to classes, while we have to determine the number of clusters.   


\vskip 2mm

\noindent {\bf Keywords}. sparse estimation, clustering variables, ALS


%\section{ First-level heading}
%The C98 head 1 style leaves a half-line spacing below a
%first-level heading. There should be one blank line above
%a first-level heading.
%        
%\subsection { Second-level heading}
%There should also be one blank line above a second- or
%third-level heading (but no extra space below them).
%
%Do not intent the first paragraph following a heading.
%Second and subsequent paragraphs are indented by one Tab
%character (= 3 mm). If footnotes are used, they should be
%placed at the foot of the page\footnote{ Footnotes are separated
%from the text by a blank line and a printed line of length 3.5 cm.
%They should be printed in 9-point Times Roman in single line spacing.}.
%        
%\subsubsection { Third-level heading}
%Please specify references using the conventions
%illustrated below. Each should begin on a new line, and
%second and subsequent lines should be on the same page
%indented by 3 mm.

\subsection*{References}

\begin{description}

\item
Pietersz, R., and Groenen, J.F (2004).
\textit{Rank Reduction of Correlation Matrices by Majorization.}
Quant.Finance, {\bf 4}: 649--662.

\item
Simon, D., and Abell, J. (2010).
\textit{Majorization Algorithm for Constrained Correlation Matrix Approximation},
Linear Algebra and its Apprications,, {\bf 432}, 1152-1164.

\end{description}

\end{document}





