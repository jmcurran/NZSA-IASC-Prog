\documentclass[12pt]{article}
% \documentstyle{iascars2017}

% \usepackage{iascars2017}

\pagestyle{myheadings} 
\pagenumbering{arabic}
\topmargin 0pt \headheight 23pt \headsep 24.66pt
%\topmargin 11pt \headheight 12pt \headsep 13.66pt
\parindent = 3mm 


\begin{document}


\begin{flushleft}


{\LARGE\bf Estimation of animal density from acoustic detections}


\vspace{1.0cm}

Ben C. Stevenson$^1$ and David L. Borchers$^2$

\begin{description}

\item $^1 \;$ Department of Statistics, University of Auckland, New Zealand

\item $^2 \;$ School of Mathematics and Statistics, University of St Andrews, United Kingdom

\end{description}

\end{flushleft}

%  ***** ADD ENOUGH VERTICAL SPACE HERE TO ENSURE THAT THE *****
%  ***** ABSTRACT (OR MAIN TEXT) STARTS 5 CM BELOW THE TOP *****

\vspace{0.75cm}

\noindent {\bf Abstract}. Estimating the density of animal populations
is of central importance in ecology, with practical applications that
affect decision making in the fields of wildlife management,
conservation, and beyond. For species that vocalise, surveys using
acoustic detectors such as microphones, hydrophones, or human
observers can be vastly cheaper than traditional surveys that
physically capture or visually detect animals. In this talk I describe
a spatial capture-recapture approach to estimate animal density from
acoustic surveys and present a software implementation in the R
package \texttt{ascr}, with examples applied to populations of frogs,
gibbons, and whales.

\vskip 2mm

\noindent {\bf Keywords}. Ecological statistics, mark-recapture, point
process

\end{document}





