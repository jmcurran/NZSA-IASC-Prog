\documentclass[12pt]{article}
% \documentstyle{iascars2017}

% \usepackage{iascars2017}

\pagestyle{myheadings} 
\pagenumbering{arabic}
\topmargin 0pt \headheight 23pt \headsep 24.66pt
%\topmargin 11pt \headheight 12pt \headsep 13.66pt
\parindent = 3mm 


\begin{document}


\begin{flushleft}

{\LARGE\bf Performance of Bayesian Credible Interval for 
Binomial Proportion using Logit Transformation}


\vspace{1.0cm}

Toru Ogura$^1$ and Takemi Yanagimoto$^2$

\begin{description}

\item $^1 \;$ Clinical Research Support Center, Mie University Hospital, 2-174, Edobashi, Tsu, Mie, Japan
\item $^2 \;$ Institute of Statistical Mathematics, 10-3, Midori-cho, Tachikawa, Tokyo, Japan

\end{description}

\end{flushleft}

%  ***** ADD ENOUGH VERTICAL SPACE HERE TO ENSURE THAT THE *****
%  ***** ABSTRACT (OR MAIN TEXT) STARTS 5 CM BELOW THE TOP *****

\vspace{0.75cm}

\noindent {\bf Abstract}. 
The confidence or the credible interval of the binomial proportion $p$ is one of most widely employed statistical analysis methods, and a variety of methods have been proposed.
The Bayesian credible interval attracts recent researches' attentions.
One of the promising methods is the highest posterior density (HPD) interval, which implies the shortest possible interval enclosing $100(1-\alpha)$\% of the probability density function.
The HPD interval is often used because it is narrow compared to other credible intervals.
However, the HPD interval has some drawbacks when the binomial proportion is a small.
To dissolve them, we calculate first a credible interval by the HPD interval of the logit transformed parameter, $\theta=\log\{p/(1-p)\}$, instead of $p$.
Note that $\theta$ and $p$ are the canonical and the mean parameters of the binomial distribution in the exponential family, respectively.
Writing the HPD interval of $\theta$ as $(\theta_{l}, \theta_{u})$, we define the proposed credible interval of $p$ as $(p_{l}, p_{u})= \big( e^{\theta_{l}} / ( 1+e^{\theta_{l}} ), \,  e^{\theta_{u}}/(1+e^{\theta_{u}}) \big)$.
It is explored in depth, and numerical comparison studies are conducted to confirm its favorable performance, especially when the observed number is small, such as 0 or 1.
Practical datasets are analyzed to examine the potential usefulness for applications in medical fields.

\vskip 2mm

\noindent {\bf Keywords}.
Bayesian credible interval, binomial proportion, highest posterior density interval, logit transformation, zero count

%\section{ First-level heading}
%The C98 head 1 style leaves a half-line spacing below a
%first-level heading. There should be one blank line above
%a first-level heading.
%        
%\subsection { Second-level heading}
%There should also be one blank line above a second- or
%third-level heading (but no extra space below them).
%
%Do not intent the first paragraph following a heading.
%Second and subsequent paragraphs are indented by one Tab
%character (= 3 mm). If footnotes are used, they should be
%placed at the foot of the page\footnote{ Footnotes are separated
%from the text by a blank line and a printed line of length 3.5 cm.
%They should be printed in 9-point Times Roman in single line spacing.}.
%        
%\subsubsection { Third-level heading}
%Please specify references using the conventions
%illustrated below. Each should begin on a new line, and
%second and subsequent lines should be on the same page
%indented by 3 mm.

\subsection*{References}

\begin{description}
\item
Newcombe, R.G. (2012).
\textit{Confidence Intervals for Proportions and Related Measures of Effect Size}. 
Florida: Chapman and Hall/CRC.
\end{description}

\end{document}





