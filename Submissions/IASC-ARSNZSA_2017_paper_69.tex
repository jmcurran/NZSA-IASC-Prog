\documentclass[12pt]{article}
% \documentstyle{iascars2017}

% \usepackage{iascars2017}

\pagestyle{myheadings} 
\pagenumbering{arabic}
\topmargin 0pt \headheight 23pt \headsep 24.66pt
%\topmargin 11pt \headheight 12pt \headsep 13.66pt
\parindent = 3mm 


\begin{document}


\begin{flushleft}


{\LARGE\bf A package for multiple precision floating-point computation on R}


\vspace{1.0cm}

Ei-ji NAKAMA$^1$ and Junji NAKANO$^2$ 

\begin{description}

\item $^1 \;$ COM-ONE Ltd.,
2-5-1 Asahidai, Nomi-city, Ishikawa 923-1211, Japan

\item $^2 \;$ The Institute of Statistical Mathematics,
10-3 Midori-cho, Tachikawa, Tokyo 190-8562, Japan


\end{description}

\end{flushleft}

%  ***** ADD ENOUGH VERTICAL SPACE HERE TO ENSURE THAT THE *****
%  ***** ABSTRACT (OR MAIN TEXT) STARTS 5 CM BELOW THE TOP *****

\vspace{0.75cm}

\noindent {\bf Abstract}. 
As recent requirements for numerical computation performed by R become larger and more complicated, 
errors from floating-point arithmetic become problematic. In R, double precision floating-point arithmetic 
is usually performed, but it may not be adequate or precise for some situations. To avoid and detect errors 
of double precision floating-point arithmetic, multiple precision arithmetic is useful. 
Several multiple precision arithmetic packages exist on R, but their abilities are limited. 
Therefore we provide another multiple precision arithmetic package Rmpenv, which can handle multiple 
precision arithmetic for real and complex numbers, matrix product and inversion, etc. 
We also provide a syntactic sugar to make easy the multiple precision computation on R. 
We utilize a free and open source MPACK library for multiple precision arithmetic and linear algebra computation.
\vskip 2mm

\noindent {\bf Keywords}.
Double precision, floating-point arithmetic, MPACK 


%\section{ First-level heading}
%The C98 head 1 style leaves a half-line spacing below a
%first-level heading. There should be one blank line above
%a first-level heading.
%        
%\subsection { Second-level heading}
%There should also be one blank line above a second- or
%third-level heading (but no extra space below them).
%
%Do not intent the first paragraph following a heading.
%Second and subsequent paragraphs are indented by one Tab
%character (= 3 mm). If footnotes are used, they should be
%placed at the foot of the page\footnote{ Footnotes are separated
%from the text by a blank line and a printed line of length 3.5 cm.
%They should be printed in 9-point Times Roman in single line spacing.}.
%        
%\subsubsection { Third-level heading}
%Please specify references using the conventions
%illustrated below. Each should begin on a new line, and
%second and subsequent lines should be on the same page
%indented by 3 mm.

%\subsection*{References}
%
%\begin{description}
%
%\item
%Barnett, J.A., Payne, R.W. and Yarrow, D. (1990).
%\textit{Yeasts: Characteristics and identification: Second Edition.}
%Cambridge: Cambridge University Press.
%
%\item
%(ed.) Barnett, V., Payne, R. and Steiner, R. (1995).
%\textit{Agricultural Sustainability: Economic, Environmental and
%Statistical Considerations}. Chichester: Wiley.
%
%\item
%Payne, R.W. (1997).
%\textit{Algorithm AS314 Inversion of matrices Statistics},
%\textbf{46}, 295--298.
%
%\item
%Payne, R.W. and Welham, S.J. (1990).
%A comparison of algorithms for combination of information in generally
%balanced designs.
%In: \textit{COMPSTAT90 Proceedings in Computational Statistics}, 297--302.
%Heidelberg: Physica-Verlag.
%
%\end{description}

\end{document}





