\documentclass[12pt]{article}
% \documentstyle{iascars2017}

% \usepackage{iascars2017}

\pagestyle{myheadings} 
\pagenumbering{arabic}
\topmargin 0pt \headheight 23pt \headsep 24.66pt
%\topmargin 11pt \headheight 12pt \headsep 13.66pt
\parindent = 3mm 


\begin{document}


\begin{flushleft}


{\LARGE\bf Two stage approach  \\ to data-driven subgroup identification in clinical trials}

\vspace{1.0cm}
%\begin{center}
Toshio Shimokawa$^1$ and Kensuke Tanioka$^1$
%\end{center}

\begin{description}
\item $^1 \;$ School of Medicine, Wakayama-Medical University, Japan
%\item $^2 \;$ Clinical Study Support Center,Wakayama-Medical University Hospital, Japan
\end{description}

\end{flushleft}

%  ***** ADD ENOUGH VERTICAL SPACE HERE TO ENSURE THAT THE *****
%  ***** ABSTRACT (OR MAIN TEXT) STARTS 5 CM BELOW THE TOP *****

\vspace{0.1cm}

\noindent {\bf Abstract}. 
A personalized medicine have been improved through the statistic analysis
of Big data such as registry data.
In these researches, subgroup identification analysis have been focused on.
The purpose of the analysis is detecting subgroup  such that the efficacy of the 
medical treatment is effective based on predictive factors for the treatment.  

Foster et al., (2011) proposed the subgroup identification method based on two stage approach,
called Virtual Twins (VT) method.
In the first stage of VT, the difference of treatment effect between treatment group and control group is
estimated by Random Forest.
In the second stage, responders are identified by using CART, where the estimated 
these differences are set as the predictor variables. 

However, 
the prediction accuracy of  RandomForest tends to be lower than that of Boosting.
Therefore, generalized boosted model (Ridgeway, 2006) is adopted in the first step. 
In addition to that, the number of rules tend to be large in the second step when CART is used.
In this paper, we adopt a priori algorithm as the same way of SIDES(Lipkovich et al., 2011).  

\vskip 2mm

\noindent {\bf Keywords}.
A priori algorithm, boosting, personalized medicine 


%\section{ First-level heading}
%The C98 head 1 style leaves a half-line spacing below a
%first-level heading. There should be one blank line above
%a first-level heading.
%        
%\subsection { Second-level heading}
%There should also be one blank line above a second- or
%third-level heading (but no extra space below them).
%
%Do not intent the first paragraph following a heading.
%Second and subsequent paragraphs are indented by one Tab
%character (= 3 mm). If footnotes are used, they should be
%placed at the foot of the page\footnote{ Footnotes are separated
%from the text by a blank line and a printed line of length 3.5 cm.
%They should be printed in 9-point Times Roman in single line spacing.}.
%        
%\subsubsection { Third-level heading}
%Please specify references using the conventions
%illustrated below. Each should begin on a new line, and
%second and subsequent lines should be on the same page
%indented by 3 mm.
\vspace{-0.5cm}
\subsection*{References}

\baselineskip 5mm
\begin{description}
\begin{small}
\item
Forster, J.C., Taylor, J.M.G and Ruberg, S.J. (2011).
\textit{Subgroup identification from randomized clinical trial data.}
Stat.Med, {\bf 30}, 2867-2880.
%Cambridge: Cambridge University Press.

\item
Lipkovich, I., Dmitrienko, A., Denne, J. and Enas, G. (2011). 
\textit{Subgroup identification based on
differential effect search-recursive partitioning method for establishing response to treatment in
patient subpopulations}. Stat.Med, {\bf 30}, 2601-2880.

\item
Ridgeway, G. (2006).Gbm: Generalized boosted regression models. R package version 1.5-7. Available at
\verb|http://www.i-pensieri.com/gregr/gbm.shtml.|

%\item
%(ed.) Barnett, V., Payne, R. and Steiner, R. (1995).
%\textit{Agricultural Sustainability: Economic, Environmental and
%Statistical Considerations}. Chichester: Wiley.

%\item
%Payne, R.W. (1997).
%\textit{Algorithm AS314 Inversion of matrices Statistics},
%\textbf{46}, 295--298.

%\item
%Payne, R.W. and Welham, S.J. (1990).
%A comparison of algorithms for combination of information in generally
%balanced designs.
%In: \textit{COMPSTAT90 Proceedings in Computational Statistics}, 297--302.
%Heidelberg: Physica-Verlag.
\end{small}
\end{description}

\end{document}





