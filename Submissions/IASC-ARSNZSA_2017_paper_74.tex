\documentclass[12pt]{article}
% \documentstyle{iascars2017}

% \usepackage{iascars2017}

\pagestyle{myheadings} 
\pagenumbering{arabic}
\topmargin 0pt \headheight 23pt \headsep 24.66pt
%\topmargin 11pt \headheight 12pt \headsep 13.66pt
\parindent = 3mm 


\begin{document}


\begin{flushleft}


{\LARGE\bf Adjusted adaptive index model for binary response}

\vspace{1.0cm}
%\begin{center}
Ke Wan$^1$,  Kensuke Tanioka$^1$, Kun Yan$^2$ and Toshio Shimokawa$^1$
%\end{center}

\begin{description}
\item $^1 \;$ School of Medicine, Wakayama-Medical University, Japan
\item $^2 \;$ College of Traffic \& Transportation, Southwest Jiaotong University, China
\end{description}

\end{flushleft}

%  ***** ADD ENOUGH VERTICAL SPACE HERE TO ENSURE THAT THE *****
%  ***** ABSTRACT (OR MAIN TEXT) STARTS 5 CM BELOW THE TOP *****

\vspace{0.75cm}

\noindent {\bf Abstract}. 

In questionnaire surveys,  
multiple regression analysis is usually used to %reveal 
evaluate influence factors.
%as a statical analysis.
In addition to that, 
data mining methods such as
Classification and Regression Trees (Breiman et al., 1984) are also used. 
In the research for tourism studies,
it is difficult to contribute the policies 
for landscape or buildings from the results.
In this paper, we call these factors `` uncontrollable exploratory variables".
On the other hands, 
the polices for amounts of garbages or inhabitant consciousness
can be contributed from the results.
We call these factors ``controllable exploratory variables".
%
The purpose of this report is grading for each subject which is conducted based on 
controllable exploratory variables with adjusting the effects of
uncontrollable exploratory variables.
%
Concretely, we modified the AIM method (Tian and Tibshirani, 2010)  and 
conduct gradings based on the sum of the production rules for 
controllable exploratory variables with adjusting the effects of uncontrollable exploratory variables.

\vskip 2mm

\noindent {\bf Keywords}.
logistic regression, production rule, grading 


%\vspace{-0.5cm}



\subsection*{References}

\baselineskip 5mm
\begin{description}
%\begin{small}
\item
Breiman, L., Friedman, J.H., Olshen, R.A. and Stone, C.J. (1984).
\textit{Classification and Regression Trees}. Wadsworth.

\item
Tian, L., and Tibshirani, R. (2011).
\textit{Adaptive index models for marker-based risk stratification.}
Biostatistics, {\bf 12}, 68--86.
%Cambridge: Cambridge University Press.

%\item
%Lipkovich, I., Dmitrienko, A., Denne, J. and Enas, G. (2011). 
%\textit{Subgroup identification based on
%differential effect search-recursive partitioning method for establishing response to treatment in
%patient subpopulations}. Stat.Med, {\bf 30}, 2601-2880.

%\item
%Ridgeway, G. (2006).Gbm: Generalized boosted regression models. R package version 1.5-7. Available at
%\verb|http://www.i-pensieri.com/gregr/gbm.shtml.|

%\item
%(ed.) Barnett, V., Payne, R. and Steiner, R. (1995).
%\textit{Agricultural Sustainability: Economic, Environmental and
%Statistical Considerations}. Chichester: Wiley.

%\item
%Payne, R.W. (1997).
%\textit{Algorithm AS314 Inversion of matrices Statistics},
%\textbf{46}, 295--298.

%\item
%Payne, R.W. and Welham, S.J. (1990).
%A comparison of algorithms for combination of information in generally
%balanced designs.
%In: \textit{COMPSTAT90 Proceedings in Computational Statistics}, 297--302.
%Heidelberg: Physica-Verlag.
%\end{small}
\end{description}

\end{document}





