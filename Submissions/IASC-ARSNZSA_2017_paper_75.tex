\documentclass[12pt]{article}
% \documentstyle{iascars2017}

% \usepackage{iascars2017}

\pagestyle{myheadings} 
\pagenumbering{arabic}
\topmargin 0pt \headheight 23pt \headsep 24.66pt
%\topmargin 11pt \headheight 12pt \headsep 13.66pt
\parindent = 3mm 


\begin{document}


\begin{flushleft}


{\LARGE\bf Dimension Reduction for Classification of High-dimensional Data by Stepwise SVM }


\vspace{1.0cm}

Elizabeth P. Chou$^1$ and Tzu-Wei Ko$^1$

\begin{description}

\item $^1 \;$ Department of Statistics, National Chengchi University, Taiwan 


\end{description}

\end{flushleft}

%  ***** ADD ENOUGH VERTICAL SPACE HERE TO ENSURE THAT THE *****
%  ***** ABSTRACT (OR MAIN TEXT) STARTS 5 CM BELOW THE TOP *****

\vspace{0.75cm}

\noindent {\bf Abstract}. The purpose of this study is to build a simple and intuitive wrapper method, stepwise SVM, for reducing dimension and classification of large p small n datasets. The method employs a suboptimum search procedure to determine the best subset of variables for classification. The proposed method is compared with other dimension reduction methods, such as Pearson product moment correlation coefficient (PCCs), Recursive Feature Elimination based on Random Forest (RF-RFE), and Principal Component Analysis (PCA) by using five gene expression datasets. In this study, we show that stepwise SVM can effectively select the important variables and perform well in prediction. Moreover, the predictions of reduced datasets from stepwise SVM are better than that of the unreduced datasets. Compared with other methods, the performance of stepwise SVM is more stable than PCA and RF-RFE but it is difficult to tell the difference in performance from PCCs. In conclusion, stepwise SVM can effectively eliminate the noise in data and improve the prediction accuracy.

\vskip 2mm

\noindent {\bf Keywords}.
Stepwise SVM, Dimension reduction, Feature selection, High-dimension


%\section{ First-level heading}
%The C98 head 1 style leaves a half-line spacing below a
%first-level heading. There should be one blank line above
%a first-level heading.
%        
%\subsection { Second-level heading}
%There should also be one blank line above a second- or
%third-level heading (but no extra space below them).
%
%Do not intent the first paragraph following a heading.
%Second and subsequent paragraphs are indented by one Tab
%character (= 3 mm). If footnotes are used, they should be
%placed at the foot of the page\footnote{ Footnotes are separated
%from the text by a blank line and a printed line of length 3.5 cm.
%They should be printed in 9-point Times Roman in single line spacing.}.
%        
%\subsubsection { Third-level heading}
%Please specify references using the conventions
%illustrated below. Each should begin on a new line, and
%second and subsequent lines should be on the same page
%indented by 3 mm.


\end{document}





