\documentclass[12pt]{article}
% \documentstyle{iascars2017}

% \usepackage{iascars2017}

\pagestyle{myheadings}
\pagenumbering{arabic}
\topmargin 0pt \headheight 23pt \headsep 24.66pt
%\topmargin 11pt \headheight 12pt \headsep 13.66pt
\parindent = 3mm


\begin{document}


\begin{flushleft}


{\LARGE\bf Real-time transit network modelling for improved
arrival time predictions}


\vspace{1.0cm}

Tom Elliott and Thomas Lumley

\begin{description}

\item Department of Statistics, University of Auckland,
Auckland, New Zealand


\end{description}

\end{flushleft}

%  ***** ADD ENOUGH VERTICAL SPACE HERE TO ENSURE THAT THE *****
%  ***** ABSTRACT (OR MAIN TEXT) STARTS 5 CM BELOW THE TOP *****

\vspace{0.75cm}

\noindent {\bf Abstract}. %
The growing availability of GPS tracking devices means that %
public transport passengers can now check on the real-time location %
of their bus from their mobile phone, %
helping them to decide when to leave home, %
and once at the stop, how long until the bus arrives. %
A side effect of this technology is that %
statistical models using vehicle location data %
to predict arrival times have taken a ``back seat'' %
in preference for methods that are simpler and faster, %
but less robust. %
Auckland Transport, who operate our local public transport network, %
demonstrate this: %
the estimated arrival time (ETA) of a bus at a stop is simply %
the time until scheduled arrival, %
plus the delay at the bus' most recently visited stop. %
The most evident problem with this approach is that %
intermediate stops, traffic lights, and road congestion---%
all of which affect ETAs---are not considered. %
We have been developing a modelling framework consisting of %
(1) a vehicle state model to infer parameters, such as speed, %
from a sequence of GPS positions; %
(2) a transit network model that uses information from the vehicle model %
to estimate traffic conditions along roads in the network; %
and (3) a predictive model combining vehicle and transit network %
states to predict arrival times. %
Since multimodality is common---%
for example a bus may or may not stop at a bus stop or traffic lights---%
we are using a particle filter to estimate vehicle state, %
which makes no assumptions about the shape of the distribution, %
and allows for a more intuitive likelihood function. %
While this provides a very flexible framework, %
it is also a computationally intensive one, %
so computational demands need to be considered %
to ensure it will be viable as a real-time application %
for providing passengers with improved, and hopefully reliable, %
arrival time information.




\vskip 2mm

\noindent {\bf Keywords}.
transit, real-time, particle filter



\end{document}
