\documentclass[12pt]{article}
% \documentstyle{iascars2017}

% \usepackage{iascars2017}

\pagestyle{myheadings} 
\pagenumbering{arabic}
\topmargin 0pt \headheight 23pt \headsep 24.66pt
%\topmargin 11pt \headheight 12pt \headsep 13.66pt
\parindent = 3mm 


\begin{document}


\begin{flushleft}


{\LARGE\bf Ranking Potential Shoplifters in Real Time}


\vspace{1.0cm}

Barry McDonald 

 \vspace{0.5cm}

Institute of Natural and Mathematical Sciences, Massey University, Albany, Auckland, NZ

 

\end{flushleft}

%  ***** ADD ENOUGH VERTICAL SPACE HERE TO ENSURE THAT THE *****
%  ***** ABSTRACT (OR MAIN TEXT) STARTS 5 CM BELOW THE TOP *****

\vspace{2 cm}

\noindent {\bf Abstract}. A company with a focus on retail crime prevention brought to MINZ (Mathematics in Industry in New Zealand)  the task of ``{\it Who is most likely to offend in my store, now}''.  The company supplied an anonymised set of data on incidents and offenders. The task, for the statisticians and mathematicians involved, was to try to find ways to use the data to nominate, say, the top ten likely offenders for any particular store and any particular time, using up-to-the-minute information (real time).  The problem was analogous to finding a regression model when every row of data has response identically 1 (an incident), and for many places and times there is no data.    This talk will describe how the problem was tackled.


\vskip 6 mm

\noindent {\bf Keywords}.
Retail crime, ranking, ZINB, regression, real time



\end{document}





