\documentclass[12pt]{article}
% \documentstyle{iascars2017}

% \usepackage{iascars2017}

\pagestyle{myheadings}
\pagenumbering{arabic}
\topmargin 0pt \headheight 23pt \headsep 24.66pt
%\topmargin 11pt \headheight 12pt \headsep 13.66pt
\parindent = 3mm


\begin{document}


\begin{flushleft}


{\LARGE\bf A Computational Efficient Algorithm for Star-Shape
Change-Curve Detection in Random Fields with Heavy-Tailed
Distribution}


\vspace{1.0cm}

Tsung-Lin Cheng$^1$ and Jheng-Ting Wang$^2$

\begin{description}

\item $^1 \;$ Department of Mathematics and Graduate Institute of
Statistic and Information, National Changhua University of
Education, Changhua City, Taiwan, Republic of China

\item $^2 \;$ Department of Mathematics, National Changhua
University of Education, Changhua City, Taiwan, Republic of China
\end{description}

\end{flushleft}

%  ***** ADD ENOUGH VERTICAL SPACE HERE TO ENSURE THAT THE *****
%  ***** ABSTRACT (OR MAIN TEXT) STARTS 5 CM BELOW THE TOP *****

\vspace{0.75cm}

\noindent {\bf Abstract}. One of the difficulties in detecting the
change boundary in a random field
 is the implementation, especially when the random disturbances have
 heavy-tailed distributions. Thank to the gearing of the digital technology, a huge amount of image
 data can be easily obtained and analyzed.
 So far, in the case of star-shape boundaries (e.g. circular or  elliptical), some well-known methods
 which deal with random fields in Cartesian coordinate cannot be directly
 applied. In particular, among the existed approaches, none is
 computationally efficient in detecting a change boundary in a
 random field. Based on Cheng's work ([1]), we propose a modified computationally efficient method to detect the star-shaped change boundaries
 in a polar coordinated random field with heavy-tailed
 distribution.

\vskip 2mm

\noindent {\bf Keywords}. change-curve detection, heavy-tailed
random fields

%\section{ First-level heading}
%The C98 head 1 style leaves a half-line spacing below a
%first-level heading. There should be one blank line above
%a first-level heading.
%
%\subsection { Second-level heading}
%There should also be one blank line above a second- or
%third-level heading (but no extra space below them).
%
%Do not intent the first paragraph following a heading.
%Second and subsequent paragraphs are indented by one Tab
%character (= 3 mm). If footnotes are used, they should be
%placed at the foot of the page\footnote{ Footnotes are separated
%from the text by a blank line and a printed line of length 3.5 cm.
%They should be printed in 9-point Times Roman in single line spacing.}.
%
%\subsubsection { Third-level heading}
%Please specify references using the conventions
%illustrated below. Each should begin on a new line, and
%second and subsequent lines should be on the same page
%indented by 3 mm.

\subsection*{References}

\begin{description}

\item[1]{\em Cheng, Tsung-Lin.} (2009). An efficient algorithm for
estimating a change-point. {\em Stat. Probab. Lett.}, {\bf79},
559-565.

\end{description}

\end{document}
