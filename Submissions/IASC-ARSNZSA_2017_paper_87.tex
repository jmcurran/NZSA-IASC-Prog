\documentclass[12pt]{article}
% \documentstyle{iascars2017}

% \usepackage{iascars2017}

\pagestyle{myheadings} 
\pagenumbering{arabic}
\topmargin 0pt \headheight 23pt \headsep 24.66pt
%\topmargin 11pt \headheight 12pt \headsep 13.66pt
\parindent = 3mm 


\begin{document}


\begin{flushleft}


{\LARGE\bf Symbolic data analytical approach to unauthorized-access logs}

\vspace{1.0cm}

Hiroyuki MINAMI and Masahiro MIZUTA

\begin{description}

\item Information Initiative Center, Hokkaido University, Sapporo 060-0811
  JAPAN
\end{description}

\end{flushleft}

%  ***** ADD ENOUGH VERTICAL SPACE HERE TO ENSURE THAT THE *****
%  ***** ABSTRACT (OR MAIN TEXT) STARTS 5 CM BELOW THE TOP *****

\vspace{0.75cm}

\noindent {\bf Abstract}. We have been annoyed by tons of unwilling
accesses in many Internet applications including SSH (Secure SHell) known
as a typical remote access tool and E-mail delivery protocols.
An attacked server put a report according to the configuration and the log
files have grown day by day.

Bad accesses might be caused by computer virus and so-called {\em zombie},
a hi-jacked computer. We assume that the actions would have their
own trends. For example, we sometimes find that a few attacks come
simultaneously from only 1 site, however, we also find several attacks
from a set of the sites within 1 minute or 1 day. The IP-Addresses might
be variable, however, within the assigned range. It suggests that the
victimizer is just one but gets an IP-Address allocation so many times.

To analyze the log files and give an interpretation to them, we introduce
Symbolic Data Analysis (SDA) to adopt its main idea {\em concept}.
If we configure an appropriate {\em concept} whose elements ({\em
  individuals} in SDA) are IP-Address, port-numbers and attack time span,
we can reveal some relationship between {\em concepts} and classify them
into perspective. The results would give us some useful information to
protect our Internet environment.

We discuss how we get them and the interpretation appropriately through
some practical examples.

\vskip 2mm

\noindent {\bf Keywords}.
Invalid network access, Firewall, Massive Data Analysis

%\section{ First-level heading}
%The C98 head 1 style leaves a half-line spacing below a
%first-level heading. There should be one blank line above
%a first-level heading.
%        
%\subsection { Second-level heading}
%There should also be one blank line above a second- or
%third-level heading (but no extra space below them).
%
%Do not intent the first paragraph following a heading.
%Second and subsequent paragraphs are indented by one Tab
%character (= 3 mm). If footnotes are used, they should be
%placed at the foot of the page\footnote{ Footnotes are separated
%from the text by a blank line and a printed line of length 3.5 cm.
%They should be printed in 9-point Times Roman in single line spacing.}.
%        
%\subsubsection { Third-level heading}
%Please specify references using the conventions
%illustrated below. Each should begin on a new line, and
%second and subsequent lines should be on the same page
%indented by 3 mm.

\subsection*{References}

\begin{description}
\item
Collins, M. (2014). \textit{Network Security Through Data Analysis}. O'Reilly.

\item
Minami, H. and Mizuta, M. (2016). A study on the Analysis of the refused
logs by Internet Firewall. \textit {Proceedings of 2016 International
  Conference for JSCS 30th Anniversary in Seattle}.

\end{description}

\end{document}





