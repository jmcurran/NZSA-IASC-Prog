\documentclass[12pt]{article}
% \documentstyle{iascars2017}

% \usepackage{iascars2017}

\pagestyle{myheadings}
\pagenumbering{arabic}
\topmargin 0pt \headheight 23pt \headsep 24.66pt
%\topmargin 11pt \headheight 12pt \headsep 13.66pt
\parindent = 3mm


\begin{document}


\begin{flushleft}


{\LARGE\bf Deep Learning High-Dimensional Covariance Matrices}


\vspace{1.0cm}

Philip L.H. Yu$^1$ and Yaohua Tang$^1$

\begin{description}

\item $^1 \;$ Department of Statistics and Actuarial Science, The University of Hong Kong,
Hong Kong, China


\end{description}

\end{flushleft}

%  ***** ADD ENOUGH VERTICAL SPACE HERE TO ENSURE THAT THE *****
%  ***** ABSTRACT (OR MAIN TEXT) STARTS 5 CM BELOW THE TOP *****

\vspace{0.75cm}

\noindent {\bf Abstract}. Modeling and forecasting covariance matrices of asset returns play a crucial role in finance. The availability of high frequency intraday data enables the modeling of the realized covariance matrix directly. However, most models in the literature depend on strong structural assumptions and they also suffer from the curse of dimensionality. To solve the problem, we propose a deep learning model which treats each realized covariance matrix as an image. The network structure is designed with simplicity in mind, and yet provides superior accuracy compared with several advanced statistical methods. The model could handle both low-dimensional and high-dimensional realized covariance matrices.

\vskip 2mm

\noindent {\bf Keywords}.
Deep learning, Realized covariance matrix, Convolutional neural network


%\section{ First-level heading}
%The C98 head 1 style leaves a half-line spacing below a
%first-level heading. There should be one blank line above
%a first-level heading.
%
%\subsection { Second-level heading}
%There should also be one blank line above a second- or
%third-level heading (but no extra space below them).
%
%Do not intent the first paragraph following a heading.
%Second and subsequent paragraphs are indented by one Tab
%character (= 3 mm). If footnotes are used, they should be
%placed at the foot of the page\footnote{ Footnotes are separated
%from the text by a blank line and a printed line of length 3.5 cm.
%They should be printed in 9-point Times Roman in single line spacing.}.
%
%\subsubsection { Third-level heading}
%Please specify references using the conventions
%illustrated below. Each should begin on a new line, and
%second and subsequent lines should be on the same page
%indented by 3 mm.

\subsection*{References}

\begin{description}

\item
LeCun, Y., Bottou, L., Bengio, Y. and Haffner, P. (1998). 
Gradient-based learning applied to document recognition. In 
\textit{Proceedings of the IEEE}, 86, 2278--2324.

\item
Shen, K., Yao, J. and Li, W. K.(2015). Forecasting High-Dimensional Realized Volatility
Matrices Using A Factor Model. \textit{ArXiv e-prints}.

\item
Tao, M., Wang, Y., Yao, Q. and Zou, J. (2011). 
Large volatility matrix inference via combining low-frequency and high-frequency approaches. \textit{Journal of the American Statistical Association},
106, 1025--1040.


\end{description}

\end{document}





Barnett, J.A., Payne, R.W. and Yarrow, D. (1990).
\textit{Yeasts: Characteristics and identification: Second Edition.}
Cambridge: Cambridge University Press.

\item
(ed.) Barnett, V., Payne, R. and Steiner, R. (1995).
\textit{Agricultural Sustainability: Economic, Environmental and
Statistical Considerations}. Chichester: Wiley.

\item
Payne, R.W. (1997).
\textit{Algorithm AS314 Inversion of matrices Statistics},
\textbf{46}, 295--298.

\item
Payne, R.W. and Welham, S.J. (1990).
A comparison of algorithms for combination of information in generally
balanced designs.
In: \textit{COMPSTAT90 Proceedings in Computational Statistics}, 297--302.
Heidelberg: Physica-Verlag.
